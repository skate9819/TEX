\documentclass[12pt,a4paper]{scrartcl}
\usepackage[utf8]{inputenc}
\usepackage[english,russian]{babel}
\usepackage{indentfirst}
\usepackage{misccorr}
\usepackage{graphicx}
\usepackage{amsmath}
\usepackage{listings}
\usepackage{amsmath,amsfonts,amssymb,amsthm,mathtools} % AMS
\usepackage{icomma}
\usepackage{mathbbol}
\makeatletter
\renewcommand*\env@matrix[1][*\c@MaxMatrixCols c]{%
	\hskip -\arraycolsep
	\let\@ifnextchar\new@ifnextchar
	\array{#1}}
\makeatother
\newcommand{\rom}[1]{\uppercase\expandafter{\romannumeral #1\relax}}

\begin{document}
	\begin{center}	
		ИДЗ №4\\
		Григорьев Дмитрий БПМИ--163\\
		Вариант 3
	\end{center}
	\textbf{Задание 1.} \\
	$e_1 = (0, 3, 0) \\
	e_2 = (-3, 0, 2)\\
	e_3 = (2, -1, 3)\\
	e'_1 = (5, 2, 1)\\
	e'_2 = (-1, -4, 5)\\
	e'_3 = (-4, 5, -6)$,\\
	
	\noindent
	и вектор $v$, имеющий в базисе e координаты $(-2, 1, 4)$. \\
	\\
	a) \textbf{Нужно найти матрицу перехода от базиса $e$ к базису $e'$.}\\
	Пусть матрица перехода из стандартного базиса в $e$ -- это $T_1$, а $T_2$ - это матрица перехода из стандар
	тного базиса в $e'$.\\
	Тогда матрица перехода из $e$ в $e'$ будет равна $T = T_1^{-1} \cdot T_2$. \\
	$T_1$ = $
	\begin{pmatrix}
	0 & -3 & 2\\[3pt]
	3 & 0 & -1\\[3pt]
	0 & 2 & 3\\[3pt]
	\end{pmatrix}$, 
	$T_2$ = $
	\begin{pmatrix}
	5 & -1 & -4\\[3pt]
	2 & -4 & 5\\[3pt]
	1 & 5 & -6\\[3pt]
	\end{pmatrix}$\\
	Терперь найдем $T_1^{-1}$: \\
	$
	\begin{pmatrix}[ccc|ccc]
	0 & -3 & 2 & 1 & 0 & 0  \\[3pt]
	3 & 0 & -1 & 0 & 1 & 0  \\[3pt]
	0 & 2 & 3 & 0 & 0 & 1  \\[3pt]
	\end{pmatrix}
	$
	$
	\xrightarrow{\rom{1} \leftrightarrow \rom{2}}
	\begin{pmatrix}[ccc|ccc]
	3 & 0 & -1 & 0 & 1 & 0  \\[3pt]
	0 & -3 & 2 & 1 & 0 & 0  \\[3pt]
	0 & 2 & 3 & 0 & 0 & 1  \\[3pt]
	\end{pmatrix}
	$
	$
	\xrightarrow{\rom{3} - \rom{2} \cdot \frac{-2}{3}}
	\begin{pmatrix}[ccc|ccc]
	3 & 0 & -1 & 0 & 1 & 0  \\[3pt]
	0 & -3 & 2 & 1 & 0 & 0  \\[3pt]
	0 & 0 & \frac{ 13 }{ 3 } & \frac{ 2 }{ 3 } & 0 & 1  \\[3pt]
	\end{pmatrix}
	$
	$
	\xrightarrow{\rom{3} = \rom{3} / \frac{13}{3}}
	\begin{pmatrix}[ccc|ccc]
	3 & 0 & -1 & 0 & 1 & 0  \\[3pt]
	0 & -3 & 2 & 1 & 0 & 0  \\[3pt]
	0 & 0 & \frac{ 13 }{ 3 } & \frac{ 2 }{ 3 } & 0 & 1  \\[3pt]
	\end{pmatrix}
	$ 
	$
	\xrightarrow{\rom{3} = \rom{3} / \frac{13}{3}} 
	\begin{pmatrix}[ccc|ccc]
	3 & 0 & -1 & 0 & 1 & 0  \\[3pt]
	0 & -3 & 2 & 1 & 0 & 0  \\[3pt]
	0 & 0 & 1 & \frac{ 2 }{ 13 } & 0 & \frac{ 3 }{ 13 }  \\[3pt]
	\end{pmatrix}
	$ 
	$
	\xrightarrow{\rom{2} - \rom{3} \cdot 2}\\
	\begin{pmatrix}[ccc|ccc]
	3 & 0 & -1 & 0 & 1 & 0  \\[3pt]
	0 & -3 & 0 & \frac{ 9 }{ 13 } & 0 & \frac{ -6 }{ 13 }  \\[3pt]
	0 & 0 & 1 & \frac{ 2 }{ 13 } & 0 & \frac{ 3 }{ 13 }  \\[3pt]
	\end{pmatrix}
	$
	$
	\xrightarrow{\rom{1} - \rom{3} \cdot -1}
	\begin{pmatrix}[ccc|ccc]
	3 & 0 & 0 & \frac{ 2 }{ 13 } & 1 & \frac{ 3 }{ 13 }  \\[3pt]
	0 & -3 & 0 & \frac{ 9 }{ 13 } & 0 & \frac{ -6 }{ 13 }  \\[3pt]
	0 & 0 & 1 & \frac{ 2 }{ 13 } & 0 & \frac{ 3 }{ 13 }  \\[3pt]
	\end{pmatrix}
	$
	$
	\xrightarrow{\rom{2} = \rom{2} / -3}
	\begin{pmatrix}[ccc|ccc]
	3 & 0 & 0 & \frac{ 2 }{ 13 } & 1 & \frac{ 3 }{ 13 }  \\[3pt]
	0 & 1 & 0 & \frac{ -3 }{ 13 } & 0 & \frac{ 2 }{ 13 }  \\[3pt]
	0 & 0 & 1 & \frac{ 2 }{ 13 } & 0 & \frac{ 3 }{ 13 }  \\[3pt]
	\end{pmatrix}
	$
	$
	\xrightarrow{\rom{1} = \rom{1} / 3}
	\begin{pmatrix}[ccc|ccc]
	1 & 0 & 0 & \frac{ 2 }{ 39 } & \frac{ 1 }{ 3 } & \frac{ 1 }{ 13 }  \\[3pt]
	0 & 1 & 0 & \frac{ -3 }{ 13 } & 0 & \frac{ 2 }{ 13 }  \\[3pt]
	0 & 0 & 1 & \frac{ 2 }{ 13 } & 0 & \frac{ 3 }{ 13 }  \\[3pt]
	\end{pmatrix}
	$
	$
	\begin{pmatrix}[ccc|ccc]
	1 & 0 & 0 & \frac{ 2 }{ 39 } & \frac{ 1 }{ 3 } & \frac{ 1 }{ 13 }  \\[3pt]
	0 & 1 & 0 & \frac{ -3 }{ 13 } & 0 & \frac{ 2 }{ 13 }  \\[3pt]
	0 & 0 & 1 & \frac{ 2 }{ 13 } & 0 & \frac{ 3 }{ 13 }  \\[3pt]
	\end{pmatrix}
	$ \\
	Получили, что $T_1^{-1} = $
	$
	\begin{pmatrix}
	\frac{ 2 }{ 39 } & \frac{ 1 }{ 3 } & \frac{ 1 }{ 13 }  \\[3pt]
	\frac{ -3 }{ 13 } & 0 & \frac{ 2 }{ 13 }  \\[3pt]
	\frac{ 2 }{ 13 } & 0 & \frac{ 3 }{ 13 }  \\[3pt]
	\end{pmatrix}
	$, тогда 
	$T = T_1^{-1} \cdot T_2$ = 
	$
	\begin{pmatrix}
	\frac{ 2 }{ 39 } & \frac{ 1 }{ 3 } & \frac{ 1 }{ 13 }  \\[3pt]
	\frac{ -3 }{ 13 } & 0 & \frac{ 2 }{ 13 }  \\[3pt]
	\frac{ 2 }{ 13 } & 0 & \frac{ 3 }{ 13 }  \\[3pt]
	\end{pmatrix}
	$ $\cdot$
	$
	\begin{pmatrix}
		5 & -1 & -4\\[3pt]
		2 & -4 & 5\\[3pt]
		1 & 5 & -6\\[3pt]
	\end{pmatrix} =
	$
	$= 
	\begin{pmatrix}
	1 & -1 & 1\\[3pt]
	-1 & 1 & 0\\[3pt]
	1 & 1 & -2\\[3pt]
	\end{pmatrix}
	$.\\	
	б) \textbf{Нужно найти  координаты вектора $v$ в базисе $e'$.}\\
	Для начала запишем как находится вектор $v$ через $v'$: $v = T \cdot v'$, тогда $v' = T^{-1} \cdot v$.\\
	Нужно найти $T^{-1}$:\\
	$
	\begin{pmatrix}[ccc|ccc]
	1 & -1 & 1 & 1 & 0 & 0\\[3pt]
	-1 & 1 & 0 & 0 & 1 & 0\\[3pt]
	1 & 1 & -2 & 0 & 0 & 1\\[3pt]
	\end{pmatrix}
	$
	$
	\xrightarrow{\rom{2} - \rom{1} \cdot -1}
	\begin{pmatrix}[ccc|ccc]
	1 & -1 & 1 & 1 & 0 & 0  \\[3pt]
	0 & 0 & 1 & 1 & 1 & 0  \\[3pt]
	1 & 1 & -2 & 0 & 0 & 1  \\[3pt]
	\end{pmatrix}
	$
	$
	\xrightarrow{\rom{3} - \rom{1} \cdot 1}
	\begin{pmatrix}[ccc|ccc]
	1 & -1 & 1 & 1 & 0 & 0  \\[3pt]
	0 & 0 & 1 & 1 & 1 & 0  \\[3pt]
	0 & 2 & -3 & -1 & 0 & 1  \\[3pt]
	\end{pmatrix}
	$
	$
	\xrightarrow{\rom{2} \leftrightarrow \rom{3}}
	\begin{pmatrix}[ccc|ccc]
	1 & -1 & 1 & 1 & 0 & 0  \\[3pt]
	0 & 2 & -3 & -1 & 0 & 1  \\[3pt]
	0 & 0 & 1 & 1 & 1 & 0  \\[3pt]
	\end{pmatrix}
	$
	$
	\xrightarrow{\rom{3} = \rom{3} / 1}
	\begin{pmatrix}[ccc|ccc]
	1 & -1 & 1 & 1 & 0 & 0  \\[3pt]
	0 & 2 & -3 & -1 & 0 & 1  \\[3pt]
	0 & 0 & 1 & 1 & 1 & 0  \\[3pt]
	\end{pmatrix}
	$
	$
	\xrightarrow{\rom{2} - \rom{3} \cdot -3}\\
	\begin{pmatrix}[ccc|ccc]
	1 & -1 & 1 & 1 & 0 & 0  \\[3pt]
	0 & 2 & 0 & 2 & 3 & 1  \\[3pt]
	0 & 0 & 1 & 1 & 1 & 0  \\[3pt]
	\end{pmatrix}
	$
	$
	\xrightarrow{\rom{1} - \rom{3} \cdot 1}
	\begin{pmatrix}[ccc|ccc]
	1 & -1 & 0 & 0 & -1 & 0  \\[3pt]
	0 & 2 & 0 & 2 & 3 & 1  \\[3pt]
	0 & 0 & 1 & 1 & 1 & 0  \\[3pt]
	\end{pmatrix}
	$
	$
	\xrightarrow{\rom{2} = \rom{2} / 2}
	\begin{pmatrix}[ccc|ccc]
	1 & -1 & 0 & 0 & -1 & 0  \\[3pt]
	0 & 1 & 0 & 1 & \frac{ 3 }{ 2 } & \frac{ 1 }{ 2 }  \\[3pt]
	0 & 0 & 1 & 1 & 1 & 0  \\[3pt]
	\end{pmatrix}
	$
	$
	\xrightarrow{\rom{1} - \rom{2} \cdot -1}
	\begin{pmatrix}[ccc|ccc]
	1 & 0 & 0 & 1 & \frac{ 1 }{ 2 } & \frac{ 1 }{ 2 }  \\[3pt]
	0 & 1 & 0 & 1 & \frac{ 3 }{ 2 } & \frac{ 1 }{ 2 }  \\[3pt]
	0 & 0 & 1 & 1 & 1 & 0  \\[3pt]
	\end{pmatrix}
	$\\
	Получилось, что $T^{-1} = $
	$
	\begin{pmatrix}
	1 & \frac{ 1 }{ 2 } & \frac{ 1 }{ 2 }  \\[3pt]
	1 & \frac{ 3 }{ 2 } & \frac{ 1 }{ 2 }  \\[3pt]
	1 & 1 & 0  \\[3pt]
	\end{pmatrix}
	$, тогда 
	$v' = T^{-1} \cdot v = $
	$
	\begin{pmatrix}
	1 & \frac{ 1 }{ 2 } & \frac{ 1 }{ 2 }  \\[3pt]
	1 & \frac{ 3 }{ 2 } & \frac{ 1 }{ 2 }  \\[3pt]
	1 & 1 & 0  \\[3pt]
	\end{pmatrix} 
	$ $\cdot$
	$
	\begin{pmatrix}
	-2\\[3pt]
	1\\[3pt]
	4 \\[3pt]
	\end{pmatrix}
	=$
	$
	=\begin{pmatrix}
	\frac{ 1 }{ 2 } \\[3pt]
	\frac{ 3 }{ 2 } \\[3pt]
	-1 \\[3pt]
	\end{pmatrix}
	$
	\begin{flushright}	
		\textbf{Ответ: a) $ 
			\begin{pmatrix}
			1 & -1 & 1\\[3pt]
			-1 & 1 & 0\\[3pt]
			1 & 1 & -2\\[3pt]
			\end{pmatrix}
			$ \\
			б)	
			$
			\begin{pmatrix}
			\frac{ 1 }{ 2 } \\[3pt]
			\frac{ 3 }{ 2 } \\[3pt]
			-1 \\[3pt]
			\end{pmatrix}
			$
		}
	\end{flushright}
	
	
	\textbf{Задание 2.} \\	
	$a_1 = (3, 1, -1, -2, 2) \\ a_2 = (-3, 2, -2, 1, 1) \\a_3 = (-1, -2, 4, 1, 1)\\a_4 = (-1, 1, -1, 2, 1)\\a_5 = (-4, 3, -3, 1, 2)\\
	b_1 = (8, -12, 8)\\b_2 = (-12, 13, 18)\\b_3 = (2, 3, -34)\\b_4 = (-8, 8, 16)\\ b_5 = (-16, 17, 26)$\\
	а) \textbf{Нужно доказать, что существует единственное линейное отображение $\varphi: \mathbb{R}^5 \rightarrow \mathbb{R}^3$, переводящее векторы $a$ в соответствующие $b$.}\\
	Мы знаем, что для некоторого набора векторов $w_1, ..., w_n \in W$ существует единственное линейное отображение $\varphi: V \rightarrow W$ такое, что $\varphi(e_1) = w_1, ..., \varphi(e_n) = w_n$. Поэтому для того, чтобы линейное отображение $\varphi: \mathbb{R}^5 \rightarrow \mathbb{R}^3$ переводило векторы $a$ в соответствующие $b$, нужно проверить, что векторы $a_1, a_2, a_3, a_4, a_5$ являются бизисом в $\mathbb{R}^5$:\\
	$ 
	\begin{pmatrix}
	3 & 1 & -1 & -2 & 2\\[3pt]
	-3 & 2 & -2 & 1 & 1\\[3pt]
	-1 & -2 & 4 & 1 & 1\\[3pt]
	-1 & 1 & -1 & 2 & 1\\[3pt]
	-4 & 3 & -3 & 1 & 2\\[3pt]
	\end{pmatrix}
	$
	$
	\xrightarrow{\rom{2} - \rom{1} \cdot -1}
	\begin{pmatrix}
	3 & 1 & -1 & -2 & 2  \\[3pt]
	0 & 3 & -3 & -1 & 3  \\[3pt]
	-1 & -2 & 4 & 1 & 1  \\[3pt]
	-1 & 1 & -1 & 2 & 1  \\[3pt]
	-4 & 3 & -3 & 1 & 2  \\[3pt]
	\end{pmatrix}
	$
	$
	\xrightarrow{\rom{3} - \rom{1} \cdot \frac{-1}{3}}
	\begin{pmatrix}
	3 & 1 & -1 & -2 & 2  \\[3pt]
	0 & 3 & -3 & -1 & 3  \\[3pt]
	0 & \frac{ -5 }{ 3 } & \frac{ 11 }{ 3 } & \frac{ 1 }{ 3 } & \frac{ 5 }{ 3 }  \\[3pt]
	-1 & 1 & -1 & 2 & 1  \\[3pt]
	-4 & 3 & -3 & 1 & 2  \\[3pt]
	\end{pmatrix}
	$
	$
	\xrightarrow{\rom{4} - \rom{1} \cdot \frac{-1}{3}}
	\begin{pmatrix}
	3 & 1 & -1 & -2 & 2  \\[3pt]
	0 & 3 & -3 & -1 & 3  \\[3pt]
	0 & \frac{ -5 }{ 3 } & \frac{ 11 }{ 3 } & \frac{ 1 }{ 3 } & \frac{ 5 }{ 3 }  \\[3pt]
	0 & \frac{ 4 }{ 3 } & \frac{ -4 }{ 3 } & \frac{ 4 }{ 3 } & \frac{ 5 }{ 3 }  \\[3pt]
	-4 & 3 & -3 & 1 & 2  \\[3pt]
	\end{pmatrix}
	$
	$
	\xrightarrow{\rom{5} - \rom{1} \cdot \frac{-4}{3}}
	\begin{pmatrix}
	3 & 1 & -1 & -2 & 2  \\[3pt]
	0 & 3 & -3 & -1 & 3  \\[3pt]
	0 & \frac{ -5 }{ 3 } & \frac{ 11 }{ 3 } & \frac{ 1 }{ 3 } & \frac{ 5 }{ 3 }  \\[3pt]
	0 & \frac{ 4 }{ 3 } & \frac{ -4 }{ 3 } & \frac{ 4 }{ 3 } & \frac{ 5 }{ 3 }  \\[3pt]
	0 & \frac{ 13 }{ 3 } & \frac{ -13 }{ 3 } & \frac{ -5 }{ 3 } & \frac{ 14 }{ 3 }  \\[3pt]
	\end{pmatrix}
	$
	$
	\xrightarrow{\rom{3} - \rom{2} \cdot \frac{-5}{9}}
	\begin{pmatrix}
	3 & 1 & -1 & -2 & 2  \\[3pt]
	0 & 3 & -3 & -1 & 3  \\[3pt]
	0 & 0 & 2 & \frac{ -2 }{ 9 } & \frac{ 10 }{ 3 }  \\[3pt]
	0 & \frac{ 4 }{ 3 } & \frac{ -4 }{ 3 } & \frac{ 4 }{ 3 } & \frac{ 5 }{ 3 }  \\[3pt]
	0 & \frac{ 13 }{ 3 } & \frac{ -13 }{ 3 } & \frac{ -5 }{ 3 } & \frac{ 14 }{ 3 }  \\[3pt]
	\end{pmatrix}
	$
	$
	\xrightarrow{\rom{4} - \rom{2} \cdot \frac{4}{9}}
	\begin{pmatrix}
	3 & 1 & -1 & -2 & 2  \\[3pt]
	0 & 3 & -3 & -1 & 3  \\[3pt]
	0 & 0 & 2 & \frac{ -2 }{ 9 } & \frac{ 10 }{ 3 }  \\[3pt]
	0 & 0 & 0 & \frac{ 16 }{ 9 } & \frac{ 1 }{ 3 }  \\[3pt]
	0 & \frac{ 13 }{ 3 } & \frac{ -13 }{ 3 } & \frac{ -5 }{ 3 } & \frac{ 14 }{ 3 }  \\[3pt]
	\end{pmatrix}
	$
	$
	\xrightarrow{\rom{5} - \rom{2} \cdot \frac{13}{9}}
	\begin{pmatrix}
	3 & 1 & -1 & -2 & 2  \\[3pt]
	0 & 3 & -3 & -1 & 3  \\[3pt]
	0 & 0 & 2 & \frac{ -2 }{ 9 } & \frac{ 10 }{ 3 }  \\[3pt]
	0 & 0 & 0 & \frac{ 16 }{ 9 } & \frac{ 1 }{ 3 }  \\[3pt]
	0 & 0 & 0 & \frac{ -2 }{ 9 } & \frac{ 1 }{ 3 }  \\[3pt]
	\end{pmatrix}
	$
	$
	\xrightarrow{\rom{5} - \rom{4} \cdot \frac{-1}{8}}
	\begin{pmatrix}
	3 & 1 & -1 & -2 & 2  \\[3pt]
	0 & 3 & -3 & -1 & 3  \\[3pt]
	0 & 0 & 2 & \frac{ -2 }{ 9 } & \frac{ 10 }{ 3 }  \\[3pt]
	0 & 0 & 0 & \frac{ 16 }{ 9 } & \frac{ 1 }{ 3 }  \\[3pt]
	0 & 0 & 0 & 0 & \frac{ 3 }{ 8 }  \\[3pt]
	\end{pmatrix}
	$ \\
	Видно, что эти векторы линейно независимы, следовательно они являются базисом в $\mathbb{R}^5$.
	\begin{flushright}
		\textbf{ч.т.д.}
	\end{flushright}
	б) \textbf{Нужно найти базис ядра и базис образа этого линейного отображения}
	Пусть некоторый вектор $v \in Ker\varphi$. Мы можем расписать $v$ через базис: \\ $v = x_1 \cdot a_1 + x_2 \cdot a_2 + x_3 \cdot a_3 + x_4 \cdot a_4 + x_5 \cdot a_5$.\\ Распишем $\varphi(v) = x_1 \cdot b_1 + x_2 \cdot b_2 + x_3 \cdot b_3 + x_4 \cdot b_4 + x_5 \cdot b_5$.  $\varphi(v) = 0$, ведь $v \in Ker\varphi$. \\
	Найдем ФСР этой системы:\\
	
	$
	\begin{pmatrix}
	8 & -12 & 2 & -8 & -16  \\[3pt]
	-12 & 13 & 3 & 8 & 17  \\[3pt]
	8 & 18 & -34 & 16 & 26  \\[3pt]
	\end{pmatrix}
	$
	$
	\xrightarrow{\rom{2} - \rom{1} \cdot \frac{-3}{2}}
	\begin{pmatrix}
	8 & -12 & 2 & -8 & -16  \\[3pt]
	0 & -5 & 6 & -4 & -7  \\[3pt]
	8 & 18 & -34 & 16 & 26  \\[3pt]
	\end{pmatrix}
	$
	$
	\xrightarrow{\rom{3} - \rom{1} \cdot 1}
	\begin{pmatrix}
	8 & -12 & 2 & -8 & -16  \\[3pt]
	0 & -5 & 6 & -4 & -7  \\[3pt]
	0 & 30 & -36 & 24 & 42  \\[3pt]
	\end{pmatrix}
	$
	$
	\xrightarrow{\rom{3} - \rom{2} \cdot -6}
	\begin{pmatrix}
	8 & -12 & 2 & -8 & -16  \\[3pt]
	0 & -5 & 6 & -4 & -7  \\[3pt]
	0 & 0 & 0 & 0 & 0  \\[3pt]
	\end{pmatrix}
	$
	$
	\xrightarrow{\rom{2} = \rom{2} / -5}
	\begin{pmatrix}
	8 & -12 & 2 & -8 & -16  \\[3pt]
	0 & 1 & \frac{ -6 }{ 5 } & \frac{ 4 }{ 5 } & \frac{ 7 }{ 5 }  \\[3pt]
	0 & 0 & 0 & 0 & 0  \\[3pt]
	\end{pmatrix}
	$
	$
	\xrightarrow{\rom{1} - \rom{2} \cdot -12}
	\begin{pmatrix}
	8 & 0 & \frac{ -62 }{ 5 } & \frac{ 8 }{ 5 } & \frac{ 4 }{ 5 }  \\[3pt]
	0 & 1 & \frac{ -6 }{ 5 } & \frac{ 4 }{ 5 } & \frac{ 7 }{ 5 }  \\[3pt]
	0 & 0 & 0 & 0 & 0  \\[3pt]
	\end{pmatrix}
	$
	$
	\xrightarrow{\rom{1} = \rom{1} / 8}
	\begin{pmatrix}
	1 & 0 & \frac{ -31 }{ 20 } & \frac{ 1 }{ 5 } & \frac{ 1 }{ 10 }  \\[3pt]
	0 & 1 & \frac{ -6 }{ 5 } & \frac{ 4 }{ 5 } & \frac{ 7 }{ 5 }  \\[3pt]
	0 & 0 & 0 & 0 & 0  \\[3pt]
	\end{pmatrix}
	$
	$
	\xrightarrow{\rom{1} = \rom{1} \cdot 20}
	\begin{pmatrix}
	20 & 0 & -31 & 4 & 2  \\[3pt]
	0 & 1 & \frac{ -6 }{ 5 } & \frac{ 4 }{ 5 } & \frac{ 7 }{ 5 }  \\[3pt]
	0 & 0 & 0 & 0 & 0  \\[3pt]
	\end{pmatrix}
	$
	$
	\xrightarrow{\rom{2} = \rom{2} \cdot 5}
	\begin{pmatrix}
	20 & 0 & -31 & 4 & 2  \\[3pt]
	0 & 5 & -6 & 4 & 7  \\[3pt]
	0 & 0 & 0 & 0 & 0  \\[3pt]
	\end{pmatrix}
	$\\
	Имеем \\
	$$
	\left \{
		\begin{gathered}
			20x_1 - 31x_3 + 4x_4 + 2x_5 = 0\\  
			5x_2 -6x_3 + 4x_4 + 7x_5 = 0
		\end{gathered}
	\right.
	$$ \\
	Свободные переменные -- $x_3, x_4, x_5$, тогда ФСР выглядит так:\\
	$\{$
	$x_3~\cdot$
	$
	\begin{pmatrix}
	\frac{31}{20}\\[3pt]
	\frac{6}{5}\\[3pt]
	1\\[3pt]
	0\\[3pt]
	0\\[3pt]
	\end{pmatrix}
	$
	$ +~x_4~\cdot$
	$
	\begin{pmatrix}
	\frac{-1}{5}\\[3pt]
	\frac{-4}{5}\\[3pt]
	0\\[3pt]
	1\\[3pt]
	0\\[3pt]
	\end{pmatrix}
	$
	$ +~x_5~\cdot$
	$
	\begin{pmatrix}
	\frac{-1}{10}\\[3pt]
	\frac{-7}{5}\\[3pt]
	0\\[3pt]
	0\\[3pt]
	1\\[3pt]
	\end{pmatrix}
	$	
	$\}$ \\
	Тогда базис ядра будет:\\
	$v_1 = \frac{31}{20}a_1 + \frac{6}{5}a_2 + a_3$\\
	$v_2 = \frac{-1}{5}a_1 - \frac{4}{5}a_2 + a_4$\\
	$v_3 = \frac{-1}{10}a_1 - \frac{7}{5}a_2 + a_5$\\
	$v_1 = $
	$
	\begin{pmatrix}
	\frac{1}{20}\\[3pt]
	\frac{39}{20}\\[3pt]
	\frac{1}{20}\\[3pt]
	\frac{-9}{10}\\[3pt]
	\frac{53}{10}\\[3pt]
	\end{pmatrix}
	$
	$v_2 = $
	$
	\begin{pmatrix}
	\frac{4}{5}\\[3pt]
	\frac{-4}{5}\\[3pt]
	\frac{4}{5}\\[3pt]
	\frac{8}{5}\\[3pt]
	\frac{-1}{5}\\[3pt]
	\end{pmatrix}
	$
	$v_3 = $
	$
	\begin{pmatrix}
	\frac{-1}{10}\\[3pt]
	\frac{1}{10}\\[3pt]
	\frac{-1}{10}\\[3pt]
	\frac{-1}{5}\\[3pt]
	\frac{2}{5}\\[3pt]
	\end{pmatrix}
	$\\
	Размерность образа линейного отображения:\\
	$dimIm\varphi = dim\mathbb{R}^5 - dimKer\varphi = 2$\\
	Найдем базис $b_1, b_2, b_3, b_4, b_5$:\\
	$
	\begin{pmatrix}
	8 & -12 & 8  \\[3pt]
	-13 & 13 & 18  \\[3pt]
	2 & 3 & -34  \\[3pt]
	-8 & 8 & 16  \\[3pt]
	-16 & 17 & 26  \\[3pt]
	\end{pmatrix}
	$
	$
	\xrightarrow{\rom{2} - \rom{1} \cdot \frac{-3}{2}}
	\begin{pmatrix}
	8 & -12 & 8  \\[3pt]
	0 & -5 & 30  \\[3pt]
	2 & 3 & -34  \\[3pt]
	-8 & 8 & 16  \\[3pt]
	-16 & 17 & 26  \\[3pt]
	\end{pmatrix}
	$
	$
	\xrightarrow{\rom{3} - \rom{1} \cdot \frac{1}{4}}
	\begin{pmatrix}
	8 & -12 & 8  \\[3pt]
	0 & -5 & 30  \\[3pt]
	0 & 6 & -36  \\[3pt]
	-8 & 8 & 16  \\[3pt]
	-16 & 17 & 26  \\[3pt]
	\end{pmatrix}
	$
	$
	\xrightarrow{\rom{4} - \rom{1} \cdot -1}
	\begin{pmatrix}
	8 & -12 & 8  \\[3pt]
	0 & -5 & 30  \\[3pt]
	0 & 6 & -36  \\[3pt]
	0 & -4 & 24  \\[3pt]
	-16 & 17 & 26  \\[3pt]
	\end{pmatrix}
	$
	$
	\xrightarrow{\rom{5} - \rom{1} \cdot -2}
	\begin{pmatrix}
	8 & -12 & 8  \\[3pt]
	0 & -5 & 30  \\[3pt]
	0 & 6 & -36  \\[3pt]
	0 & -4 & 24  \\[3pt]
	0 & -7 & 42  \\[3pt]
	\end{pmatrix}
	$
	$
	\xrightarrow{\rom{3} - \rom{2} \cdot \frac{-6}{5}}
	\begin{pmatrix}
	8 & -12 & 8  \\[3pt]
	0 & -5 & 30  \\[3pt]
	0 & 0 & 0  \\[3pt]
	0 & -4 & 24  \\[3pt]
	0 & -7 & 42  \\[3pt]
	\end{pmatrix}
	$
	$
	\xrightarrow{\rom{4} - \rom{2} \cdot \frac{4}{5}}
	\begin{pmatrix}
	8 & -12 & 8  \\[3pt]
	0 & -5 & 30  \\[3pt]
	0 & 0 & 0  \\[3pt]
	0 & 0 & 0  \\[3pt]
	0 & -7 & 42  \\[3pt]
	\end{pmatrix}
	$
	$
	\xrightarrow{\rom{5} - \rom{2} \cdot \frac{7}{5}}
	\begin{pmatrix}
	8 & -12 & 8  \\[3pt]
	0 & -5 & 30  \\[3pt]
	0 & 0 & 0  \\[3pt]
	0 & 0 & 0  \\[3pt]
	0 & 0 & 0  \\[3pt]
	\end{pmatrix}
	$\\
	Получилось, что базис образа линейного отображения равен:\\
	$v_1' = $
	$
	\begin{pmatrix}
	8 \\[3pt]
	-12\\[3pt]
	8\\[3pt]

	\end{pmatrix}
	$
	$v_2' = $
	$
	\begin{pmatrix}
	0 \\[3pt]
	-5\\[3pt]
	30\\[3pt]
	\end{pmatrix}
	$.\\\\\\
	\noindent
	\textbf{Задание 3.} \\	
	Найдем ФСР этой системы:\\
	$
	\begin{pmatrix}
	6 & 18 & 25 & -13\\[3pt]
	2 & -14 & -20 & 14\\[3pt]
	0 & 12 & 17 & -11\\[3pt]
	\end{pmatrix}
	$
	$
	\xrightarrow{\rom{2} - \rom{1} \cdot \frac{1}{3}}
	\begin{pmatrix}
	6 & 18 & 25 & -13  \\[3pt]
	0 & -20 & \frac{ -85 }{ 3 } & \frac{ 55 }{ 3 }  \\[3pt]
	0 & 12 & 17 & -11  \\[3pt]
	\end{pmatrix}
	$
	$
	\xrightarrow{\rom{3} - \rom{2} \cdot \frac{-3}{5}}
	\begin{pmatrix}
	6 & 18 & 25 & -13  \\[3pt]
	0 & -20 & \frac{ -85 }{ 3 } & \frac{ 55 }{ 3 }  \\[3pt]
	0 & 0 & 0 & 0  \\[3pt]
	\end{pmatrix}
	$
	$
	\xrightarrow{\rom{2} = \rom{2} / -20}
	\begin{pmatrix}
	6 & 18 & 25 & -13  \\[3pt]
	0 & 1 & \frac{ 17 }{ 12 } & \frac{ -11 }{ 12 }  \\[3pt]
	0 & 0 & 0 & 0  \\[3pt]
	\end{pmatrix}
	$
	$
	\xrightarrow{\rom{1} - \rom{2} \cdot 18}
	\begin{pmatrix}
	6 & 0 & \frac{ -1 }{ 2 } & \frac{ 7 }{ 2 }  \\[3pt]
	0 & 1 & \frac{ 17 }{ 12 } & \frac{ -11 }{ 12 }  \\[3pt]
	0 & 0 & 0 & 0  \\[3pt]
	\end{pmatrix}
	$
	$
	\xrightarrow{\rom{1} = \rom{1} / 6}
	\begin{pmatrix}
	1 & 0 & \frac{ -1 }{ 12 } & \frac{ 7 }{ 12 }  \\[3pt]
	0 & 1 & \frac{ 17 }{ 12 } & \frac{ -11 }{ 12 }  \\[3pt]
	0 & 0 & 0 & 0  \\[3pt]
	\end{pmatrix}
	$
	$
	\xrightarrow{\rom{1} = \rom{1} \cdot 12}
	\begin{pmatrix}
	12 & 0 & -1 & 7  \\[3pt]
	0 & 1 & \frac{ 17 }{ 12 } & \frac{ -11 }{ 12 }  \\[3pt]
	0 & 0 & 0 & 0  \\[3pt]
	\end{pmatrix}
	$
	$
	\xrightarrow{\rom{2} = \rom{2} \cdot 12}
	\begin{pmatrix}
	12 & 0 & -1 & 7  \\[3pt]
	0 & 12 & 17 & -11  \\[3pt]
	0 & 0 & 0 & 0  \\[3pt]
	\end{pmatrix}
	$\\
	Свободные переменные -- $x_3, x_4$, тогда ФСР выглядит так:\\
	$\{$
	$x_3~\cdot$
	$
	\begin{pmatrix}
	\frac{1}{12}\\[3pt]
	\frac{-17}{12}\\[3pt]
	1\\[3pt]
	0\\[3pt]
	\end{pmatrix}
	$
	$ +~x_4~\cdot$
	$
	\begin{pmatrix}
	\frac{-7}{12}\\[3pt]
	\frac{11}{12}\\[3pt]
	0\\[3pt]
	1\\[3pt]
	\end{pmatrix}
	$
	$\}$ \\
	Теперь дополним базис ядра до базиса в $\mathbb{R}^4$:\\
	$
	\begin{pmatrix}
	0 & 0 & 1 & 0\\[3pt]
	0 & 0 & 0 & 1\\[3pt]
	\frac{1}{12} & \frac{-17}{12} & 1 & 0\\[3pt]
	\frac{-7}{12} & \frac{11}{12} & 0 & 1\\[3pt]
	\end{pmatrix}
	$ \\
	Теперь найдем $\varphi(e_1), \varphi(e_2), \varphi(e_3), \varphi(e_4)$:\\
	$\varphi(e_1)=$ 
	$
	\begin{pmatrix}
	25\\[3pt]
	-20\\[3pt]
	17\\[3pt]
	\end{pmatrix}
	$
	$\varphi(e_2)=$ 
	$
	\begin{pmatrix}
	13\\[3pt]
	14\\[3pt]
	-11\\[3pt]
	\end{pmatrix}
	$
	$\varphi(e_3)=$ 
	$
	\begin{pmatrix}
	0\\[3pt]
	0\\[3pt]
	0\\[3pt]
	\end{pmatrix}
	$
	$\varphi(e_4)=$ 
	$
	\begin{pmatrix}
	0\\[3pt]
	0\\[3pt]
	0\\[3pt]
	\end{pmatrix}
	$\\\\\\\\
	Теперь осталось дополнить базис образа до базиса $\mathbb{R}^3$:\\
	$
	\begin{pmatrix}
		25 & -20 & 17  \\[3pt]
		13 & 14 & -11\\[3pt]
		0 & 0 & 1 \\[3pt]
	\end{pmatrix}
	$\\
	Выразим $C_1$ и $C_2$:\\
	$C_1 =$ 
	$
	\begin{pmatrix}
		25 & 13 & 0  \\[3pt]
		-20 & 14 & 0\\[3pt]
		17 & -11 & 1 \\[3pt]
	\end{pmatrix}
	$
	$C_2 =$ 
	$
	\begin{pmatrix}
	0 & 0 & \frac{1}{12} & \frac{-7}{12}\\[3pt]
	0 & 0 & \frac{-17}{12} & \frac{11}{12}\\[3pt]
	1 & 0 & 1 & 0\\[3pt]
	0 & 1 & 0 & 1\\[3pt]
	\end{pmatrix}
	$ \\
	$D = C_1^{-1} \cdot A \cdot C_2$ \\
	$A = C_1 \cdot D \cdot C_2^{-1}$
	\\\\\\
	\textbf{Задание 4.} \\
	$Q(x_1, x_2, x_3) = -9x_1^2 - 13x_2^2 - 7x_3^2 + 18x_1x_2 + 12x_1x_3 - 4x_2x_3 = -9(x_1^2 - 2x_1(x_2 +\frac{2}{3}x_3) + (x_2 + \frac{2}{3}x_3)^2) + 9(x_2 + \frac{2}{3}x_3)^2 - 13x_2^2 - 7x_3^2 - 4x_2x_3 = -9(x_1 - x_2 - \frac{2}{3}x_3) ^ 2 + 9x_2^2 + 12x_2x_3 + 4x_3^2 - 13x_2^2 - 7x_3^2 - 4x_2x_3 = -9(x_1 - x_2 - \frac{2}{3}x_3) ^ 2 - 4x_2^2 + 8x_2x_3 - 3x_3^2 = -9(x_1 - x_2 - \frac{2}{3}x_3) ^ 2 - 4(x_2^2 - 2x_2x_3 + x_3^2) + x_3^2 = -9(x_1 - x_2 - \frac{2}{3}x_3) ^ 2 - 4(x_2 - x_3)^2 + x_3^2$\\
	Замена:\\
	$y_1 = 3(x_1 - x_2 - \frac{2}{3}x_3) $\\
	$y_2 = 2(x_2 - x_3)$\\
	$y_3 = x_3$\\
	А теперь выразим старые координаты через новые:\\
	$x_3 = y_3$\\
	$x_2 = \frac{1}{2}y_2 + 2y_3$\\
	$x_1 = \frac{1}{3}y_1 + \frac{1}{2}y_2 + \frac{8}{3}y_3$
\end{document}
