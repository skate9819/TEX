\documentclass[12pt,a4paper]{scrartcl}
\usepackage[utf8]{inputenc}
\usepackage[english,russian]{babel}
\usepackage{indentfirst}
\usepackage{misccorr}
\usepackage{graphicx}
\usepackage{amsmath}
\usepackage{listings}
\usepackage{amsmath,amsfonts,amssymb,amsthm,mathtools} % AMS
\usepackage{icomma}
\usepackage{mathbbol}
\makeatletter
\renewcommand*\env@matrix[1][*\c@MaxMatrixCols c]{%
	\hskip -\arraycolsep
	\let\@ifnextchar\new@ifnextchar
	\array{#1}}
\makeatother
\newcommand{\rom}[1]{\uppercase\expandafter{\romannumeral #1\relax}}

\begin{document}
	\begin{center}	
		ИДЗ №5\\
		Григорьев Дмитрий БПМИ--163\\
		Вариант 3
	\end{center}
	\textbf{Задание 1.} \\
	$Q(x_1, x_2, x_3) = x_1^2(13b - 57) + x_2^2(-7 + b) + x_3^2(5b - 19) + 2x_1x_2(15 - 3b) + 2x_1x_3(-7b + 27) + 2x_2x_3(-7 + b)$ \\
	Выпишем матрицу квадратичной формы:\\
	$$A = \begin{pmatrix}
	~13b - 57~~~~30 - 6b~~~54 - 14b\\
	30 - 6b~~~~~~b - 7~~~~~2b - 14\\
	~54 - 14b~~~~2b - 14~~~5b - 19\\
	\end{pmatrix}$$\\
	Теперь воспользуемся критерием Сильвестра. \\
	$\bullet$ Чтобы квадратичная форма была положительно определенной все угловые миноры ее матрицы должны быть положительны. Найдем все значения $b$ при которых все угловые миноры $A$ положительны:\\
	$\begin{vmatrix}
	13b - 57
	\end{vmatrix} > 0 \Rightarrow b > \frac{57}{13}$ \\
	$\begin{vmatrix}
	~13b - 57~~~~30 - 6b\\
	30 - 6b~~~~~~b - 7
	\end{vmatrix} = -23 \cdot b^2+212\cdot b-501 > 0$\\
	 Таких $b$ при кторых$-23 \cdot b^2+212 \cdot b-501 > 0$ нет, так как дискриминант меньше 0. Поэтому угловой минор третьего порядка можно не рассматривать.\\
	$\bullet$ Квадратичная форма является отрицательно определенной, тогда и только тогда, когда знаки всех угловых миноров её матрицы чередуются, причем минор порядка $1$ отрицателен. \\
	Угловой минор 2 порядка должен быть больше 0, но как выяснили в предыдущем пункте, таких $b$ при кторых$-23 \cdot b^2+212 \cdot b-501 > 0$ нет, так как дискриминант меньше 0.
	\begin{flushright}	
		\textbf{Ответ: таких $b$, при которых квадратичная форма является положительно определённой нет\\
		 таких $b$, при которых квадратичная форма является\\ отрицательно определённой нет\\} 
	\end{flushright}
	
	
	 
	\noindent
	\textbf{Задание 2.} \\	
	Задано уравнение подпространство $U$ евклидова пространства $\mathbb{R}^4~~3x_1 + x_2 + 5x_3 - x_4 = 0$.\\
	а) Нужно построить ортонормированный базис в $U$. \\
	Для начала найдем базис подпространства $U$:\\
	$$3x_1 + x_2 + 5x_3 - x_4 = 0$$
	$$x_1 = \frac{1}{3}x_4 - \frac{1}{3}x_2 - \frac{5}{3}x_3 $$
	Получаем ФСР:\\
	$$\begin{pmatrix}
	x_1\\
	x_2\\
	x_3\\
	x_4\\
	\end{pmatrix} = x_2 \begin{pmatrix}
	-\frac{1}{3}\\
	1\\
	0\\
	0\\
	\end{pmatrix} + x_3\begin{pmatrix}
	-\frac{5}{3}\\
	0\\
	1\\
	0\\
	\end{pmatrix} + x_4\begin{pmatrix}
	\frac{1}{3}\\
	0\\
	0\\
	1\\
	\end{pmatrix}$$
	\\
	Тогда базис подпространства $U$ это \\
	$$b_1(-1, 3, 0, 0), b_2(-5, 0, 3, 0), b_3(1, 0, 0, 3)$$ \\
	Теперь получим ортогональный базис с помощью процесса Грамма-Шмидта:\\
	$~~c_1 = b_1$\\
	$~~c_2 = b_2 - proj_{c_1}b_2$\\
	$~~c_3 = b_3 - proj_{c_1}b_3 - proj_{c_2}b_3$\\
	
	Где $proj_{c}b = \frac{(b, c)}{(c, c)}c$.\\
	
	\noindent
	$c_1 = (-1, 3, 0, 0)\\
	c_2 = (-\frac{9}{2}, -\frac{3}{2}, 3, 0)\\
	c_3 = (\frac{9}{35}, \frac{3}{35}, \frac{3}{7}, 3)\\$
	И далее получим ортонормированный базис(перед этим умножим $c_2$ на 2 и $c_3$ на 35):\\
	$e_1 = \frac{c_1}{|c_1|}$~~~~~~~~$e_1(-\frac{1}{\sqrt{10}}, \frac{3}{\sqrt{10}}, 0, 0)$\\
	$e_2 = \frac{c_2}{|c_2|}$~~~~~~~~$e_2(-\frac{3}{\sqrt{14}}, -\frac{1}{\sqrt{14}}, \frac{2}{\sqrt{14}}, 0)$\\
	$e_3 = \frac{c_3}{|c_3|}$~~~~~~~~$e_2(\frac{1}{2\sqrt{35}}, \frac{1}{6\sqrt{35}}, \frac{5}{6\sqrt{35}}, \frac{35}{6\sqrt{35}})$\\
	б) Для вектора $v = (0, 0, 2, 1)$ нужно найти его проекцию на $U$, его ортогональную составляющую относительно $U$ и	расстояние от него до $U$.\\
	$$v = q + w$$
	$$q = \lambda_1 c_1 + \lambda_2 c_2 + \lambda_3 c_3$$
	где $q \in U$ --  ортогональная проекция вектора $v$ относительно подпространства $U$, \\$w$ -- ортогональная составляющая этого вектора относительно подпространства $U$.\\
	
	\noindent Можно составить систему, решив которую получим $\lambda_1, \lambda_2, \lambda_3$:\\
	 $$\begin{cases}
		 (v, c_1) = \lambda_1(c_1, c_1) + \lambda_2(c_2, c_1) + \lambda_3(c_3, с_1)\\
		 (v, c_2) = \lambda_1(c_1, c_2) + \lambda_2(c_2, c_2) + \lambda_3(c_3, с_2)\\
		 (v, c_3) = \lambda_1(c_1, c_3) + \lambda_2(c_2, c_3) + \lambda_3(c_3, c_3)
	 \end{cases}$$
	 Пользуясь ортогональностью базиса $\mathbb{c}$ получим:\\
	 $\begin{cases}
	 0 = 10\lambda_1\\
	 12 = 126\lambda_2\\
	 135 = 11340\lambda_3
	 \end{cases}$ $\begin{cases}
	 \lambda_1 = 0\\
	 \lambda_2 = \frac{2}{21}\\
	 \lambda_3 = \frac{1}{84}
	 \end{cases}$\\
	 Получается, что ортогональная проекция вектора $v$ относительно подпространства $U$ равна:\\
	 $$q = \frac{2}{21} \begin{pmatrix}
	 -9\\
	 -3\\
	 6\\
	 0
	 \end{pmatrix} + \frac{1}{84} \begin{pmatrix}
	 9\\
	 3\\
	 15\\
	 105\\
	 \end{pmatrix} = \begin{pmatrix}
	 -{3}/{4}\\
	 -{1}/{4}\\
	 {3}/{4}\\
	 {5}/{4}\\
	 \end{pmatrix}
	 $$
	 Теперь найдем ортогональную состовляющую вектора $v$ относительно подпространства $U$:\\
	 $$w = v - q = \begin{pmatrix}
	 0\\
	 0\\
	 2\\
	 1\\
	 \end{pmatrix} - \begin{pmatrix}
	 -{3}/{4}\\
	 -{1}/{4}\\
	 {3}/{4}\\
	 {5}/{4}\\
	 \end{pmatrix} = \begin{pmatrix}
	 {3}/{4}\\
	 {1}/{4}\\
	 {5}/{4}\\
	 -{1}/{4}\\
	 \end{pmatrix}$$\\
	 Теперь найдем расстояние от $v$ до $U$:\\
	 $$\rho = |w| = \frac{3}{2} = 1.5$$\\
	 
	\noindent
	\textbf{Задание 3.} \\	
	 Нужно составить уравнения прямой в $\mathbb{R}^3$, параллельной плоскости $-x + 4y + z = 0$, проходящей через точку $M(1, -2, 2)$ и
	 пересекающей прямую $x = t + 4, y = 4t - 1, z = 3t + 3$. \\
	 Так как искомая прямая параллельна плоскости и проходит через точку $M$, то мы можем найти плоскость, параллельную данной и проходящую через точку $M$. Далее найдем точку пересечения прямой и найденной плоскости. И теперь нам будут изветны 2 точки через которые проходит искомая прямая.\\
	 1) Найдем плоскость, параллельную данной и проходящую через точку $M$:\\
		$$-(x - x_M) + 4(y - y_M) + (z - z_M) = 0$$
		$$-x + 1 + 4y + 8 + z - 2 = 0$$
		$$-x + 4y + z + 7 = 0$$
	2) Теперь найдем точку пересечения плоскости $-x + 4y + z + 7 = 0$ и прямой $x = t + 4, y = 4t - 1, z = 3t + 3$:\\
	$$-(t + 4) + 4(4t - 1) + (3t + 3) + 7 = 0$$
	$$18t = -2$$
	$$t = -\frac{1}{9}$$\\
	Тогда точка пересечения имеет координаты $A(3\frac{8}{9}, -1\frac{4}{9}, 2\frac{2}{3})$\\
	3)Теперь имеем две точки $A, M$, через которые проходит искомая прямая. Найдем ее уравнение:\\
	Найдем направляющий вектор прямой $MA$, $\overrightarrow{s}(2\frac{8}{9}, \frac{5}{9}, \frac{2}{3})$. Теперь у нас есть направляющий вектор искомой прямой и точки принадлежащие ей. Умножим направляющий вектор на 9 и получим $\overrightarrow{s}(26, 5, 6)$. И теперь найдем уравнение искомой прямой:\\
	$$\begin{cases}
		x = 26t + 1 \\
		 y = 5t - 2\\
		 z = 6t + 2
	\end{cases}$$\\
	\begin{flushright}	
		\textbf{Ответ:$\begin{cases}
			x = 26t + 1 \\
			y = 5t - 2\\
			z = 6t + 2
			\end{cases}$} 
	\end{flushright}
	\noindent
	\textbf{Задание 4.} \\
	Нам дан куб со стороной 3. Пусть начало координат -- точка $A$, тогда точки имеют координаты:\\
	$A(0, 0, 0), B(3, 0, 0), C(3, 0, 3), D(0, 0, 3), A'(0, 3, 0), B'(3, 3, 0), C'(3, 3, 3), D'(0, 3, 3)$, так как $F$ -- середина ребра $BB'$, то $F$ имеет координаты $F(3, 1.5, 0)$, так же точка $E$ имеет координаты $E(3, \frac{15}{7}, 0)$.\\
	Нам нужно найти угол и расстояние между прямыми $AE$ и $D'F$.\\
	$\bullet$ Найдем угол.\\
	Найдем направляющий вектор $\overrightarrow{s}$ прямой $AE$ и направляющий вектор $\overrightarrow{s_1}$ прямой $D'F$:\\
	$$\overrightarrow{s}(3, \frac{15}{7}, 0)$$ 
	$$\overrightarrow{s_1}(3, -1.5, -3)$$\\
	Теперь найдем угол по формуле $\alpha = arccos\frac{|(\overrightarrow{s}, \overrightarrow{s_1})|}{|\overrightarrow{s}| \cdot |\overrightarrow{s_1}|}$\\
	$$\alpha = arccos\frac{|9 - \frac{45}{14}|}{|\sqrt{9 + \frac{225}{49}}| \cdot |\sqrt{9 + \frac{9}{4} + 9}|} = arccos\frac{\frac{81}{14}}{|\sqrt{\frac{666}{49}}|\cdot|\sqrt{\frac{81}{4}}|} = arccos\frac{3}{\sqrt{74}}$$\\
	Получается угол между прямыми $AE$ и $D'F$ равен $arccos\frac{3}{\sqrt{74}}$.\\
	$\bullet$ Теперь найдем расстояние между прямыми $AE$ и $D'F$. Так как это скрещивающиеся прямые, то найдем плоскость, проходящую через прямую $D'F$ и параллельную прямой $AE$. Дальше нам нужно найти расстояние от какой-либо точки на прямой $AE$ до найденной плоскости. Это и будет искомое расстояни.\\
	Прямая $AE$ проходит через точку $A(0, 0, 0)$ и имеет направляющий вектор $\overrightarrow{s}(3, \frac{15}{7}, 0)$. Прямая $D'F$ имеет направляющий вектор  $\overrightarrow{s_1}(3, -1.5, -3)$. Вычислим векторное произведение $\overrightarrow{s}$ и $\overrightarrow{s_1}$:\\
	$[\overrightarrow{s}, \overrightarrow{s_1}]$ = 
	$\begin{vmatrix}
		i~~~~~ j~~~~~ k\\
		3~~~~~\frac{15}{7}~~~~ 0\\
		~3~-\frac{3}{2}~~-3\\
		
	\end{vmatrix} = \frac{(-90\cdot i + 126 \cdot j-153 \cdot k)}{14}$ \\
	Получили, что вектор нормали $\overrightarrow{n}$ = $[\overrightarrow{s}, \overrightarrow{s_1}]$ плоскости, проходящей через прямую $D'F$ и параллельной прямой $AE$ имеет координаты $\overrightarrow{n}(-\frac{90}{14}, 9, -\frac{153}{14})$\\
	Уравнение плоскости проходящей через прямую $D'F$ и параллельной прямой $AE$ выглядит так:\\
	$-\frac{90}{14}(x - 0) + 9(y - 3) -\frac{153}{14}(z - 3) = 0$\\
	$-\frac{90}{14}x + 9y -\frac{153}{14}z + \frac{81}{14}= 0$\\
	Теперь нам нужно найти расстояние от точки $A$ до найденной плоскости:\\
	$$\rho = \frac{|Ax_0 + By_0 + Cz_0 + D|}{\sqrt{A^2 + B^2 + C^2}} = \frac{\frac{81}{14}}{\sqrt{\frac{47385}{196}}} = \frac{81}{27\sqrt{65}} = \frac{3}{\sqrt{65}}$$\\
	Это и есть искомое расстояние.\\
	\begin{flushright}	
		\textbf{Ответ: угол между прямыми $AE$ и $D'F$ равен $arccos\frac{3}{\sqrt{74}}$ \\
		расстояние между прямыми $AE$ и $D'F$ равно $\frac{3}{\sqrt{65}}$ 
	} 
	\end{flushright}
	
\end{document}
