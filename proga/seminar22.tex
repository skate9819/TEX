\documentclass[12pt,a4paper]{scrartcl}
\usepackage[utf8]{inputenc}
\usepackage[english,russian]{babel}
\usepackage{indentfirst}
\usepackage{misccorr}
\usepackage{graphicx}
\usepackage{amsmath}

\begin{document}
	\begin{center}	
		Алгоритмы и структуры данных. Семинар 22.
Альфа-бета отсечение, СНМ, остовные деревья. \\
		Григорьев Дмитрий БПМИ-163\\
	\end{center}
	\textbf{Задача 2.} \\
	\\
	Асимптотика работы системы непересекающихся множеств при использовании только ранговой эвристики будет логарифмической на один запрос в среднем: $O(log~n)$. Нужно показать, что она не может быть заменена на случайное равновероятное
	подвешинивание. \\
	$-$При случайном равновероятном подвешивании мат. ожидание высоты получившегося дерева будет:
	\\
	Тут два случая: \\
	$\indent$1. Подвешиваем к дереву меньшей высоты -- тогда высота увеличится на 1\\
	$\indent$2. Подвешиваем к дереву большей высоты -- тогда высота не изменится\\
	Получится следующее:
	\\
	$E_h = 1/2(h) + 1/2(h + 1) = h + 1/2$\\
	Таким образом, мат. ожидание глубины получившегося дерева -- $\frac{n}{2}$. \\
	Получилось, что ранговая эвристика не может быть заменена на случайное равновероятное
	подвешинивание.\\
	\\
	\textbf{Задача 4.} \\
	\\
	Минимальный остов также является остовом, минимальным по произведению всех рёбер. В самом деле, если мы заменим веса всех рёбер на их логарифмы, то легко заметить, что в работе алгоритма ничего не изменится, и будут найдены те же самые рёбра. Для поиска минимального остова воспользуемся алгоритмом Прима.\\
	\\
	\textbf{Задача 5.} \\
	\\
	В этой задаче воспользуемся сортировкой подсчетом и алгоритмом Прима.\\
	У нас будет массив $edge$ от 1 до $U$ в которм будут храниться списки ребер с весами(в $edge[i]$ будут храниться ребра, которые соеденены одним концом с вершиной, входящей в текущий остов, веса $i$). Далее мы будем для каждого веса хранить указатель, начиная с которого нужно проверять ребра. \\
	Итак, при добавлении очередного ребра в остов мы добавляем все ребра исходящие из новой вершины в массив $edge$. Потом ищем нужное ребро(минимальное и входящее в остов "одним концом"), при этом сдвигая указатели, если встречаем ненужные ребра. Таким образом мы построим мин. остов. Для каждой новой вершины мы проходим по массиву $edge$, отсюда получаем $O(n \cdot U)$, так же мы пробкгаем по всем ребрам в массиве $edge$(когда сдвигаем указатели), причем по каждому ребру не более 1 раза, отсюда получаем $O(m)$. Так мы получили оценку $O(n \cdot U + m)$. \\
	\\
	\\
	\textbf{Задача 8.} \\
	\\
	а) Чтобы проверить является ли данный остов минимальным, мы можем построить за за время $O(|E|log|V|)$ другой минимальный остов алгоритмом Прима и проверить равен ли он исходному. Мы можем это сделать, так как минимальный остов в графе единственен.
\end{document}
