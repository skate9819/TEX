\documentclass[12pt,a4paper]{scrartcl}
\usepackage[utf8]{inputenc}
\usepackage[english,russian]{babel}
\usepackage{indentfirst}
\usepackage{misccorr}
\usepackage{graphicx}
\usepackage{amsmath}

\begin{document}
	\begin{center}	
		Алгоритмы и структуры данных. Семинар 31. \\
		Паросочетания в двудольных графах. \\
		Григорьев Дмитрий БПМИ-163\\
	\end{center}
	\textbf{Задача 1.} \\
	\textit{Напомним, лемма Холла утверждает, что в двудольном графе $G = (L, R, E)$ из n вершин
		в каждой доле, совершенное паросочетание существует тогда и только тогда, когда для
		любого $A \subset L, |N(A)| \geq |A|.$ Докажите эту лемму.}\\
	\\
	Решение:\\
	$\bullet$ Необходимость очевидна.\\
	если существует полное паросочетание, то для любого $A \subset  L$ выполнено $|A| \leqslant |N(A)|$. У любого подмножества вершин есть по крайней мере столько же соседей (соседи по паросочетанию).\\
	$\bullet$ Достаточность.\\
	Докажем по индукции. Будем добавлять в изначально пустое паросочетание $P$ по одному ребру и доказывать, что мы можем это сделать, если $P$ не полное. И в конце получим что $P$ — полное паросочетание.\\
	База индукции:\\
	$\indent$ Вершина из $L$ соединена хотя бы с одной вершиной из $R$. Следовательно база верна.\\
	Индукционный переход:\\
	$\indent$ Пусть после $q < n$ шагов есть паросочетание $P$. Тогда докажем, что можно добавить в $P$ вершину $x$ из $L$, не насыщенную паросочетанием $P$. \\
	Рассмотрим множество вершин $M$, где $M$ это все вершины, достижимые из $x$, если мы можем ходить из $R$ в $L$ только по ребрам из $P$, а из $L$ в $R$ по любым ребрам. Тогда в $M$ найдется вершина $y$ из $R$, не насыщенная паросочетанием $P$(так как если рассмотреть вершины из $M$ принадлежащие $L$, то для них не будет выполнено условие: $|A| \leqslant |N(A)|$). Тогда найдется путь из $x$ в $y$, который будет удлиняющим для паросочетания $P$ (так как из $R$ в $L$ мы проходили по ребрам паросочетания $P$). Тогда можем увеличить паросочетание $P$. Следовательно предположение индукции верно.\\
	\\
	\textbf{Задача 2.} \\
	\textit{Пусть у двудольного графа степени всех вершин положительны и равны. Докажите, что
		в этом графе найдётся совершенное паросочетание.}\\
	\\
	Решение:\\
	Рассмотрим произвольное множество вершин $M$, где $M$ -- вершины из $L$.\\
	 Пусть $E$ -- количество ребер, выходящих из $M$, тогда   $$E = |M| \cdot k,$$ где $k$ -- степень любой вершины. \\
	 Рассмотрим $N(M)$. Тогда $$E \leq k \cdot |N(A)|$$. \\
	 Получается, что $$|M| \cdot k \leq |N(M)| \cdot k \\
	 $$
	 $$|M| \leq |N(M)|$$
	 Получается, что для
	 любого $M \subset L, |N(M)| \geq |M|.$ Тогда по лемме Холла, так как для любого $M \subset L, |N(M)| \geq |M|$, то в этом графе найдется совершенное паросочетание.\\
	 \begin{flushright}
	 	ч.т.д.
	 	\end{flushright}
	\textbf{Задача 3.} \\
	\textit{У вас есть полный двудольный взвешенный граф. Требуется найти максимальное паросочетание, такое что вес максимального ребра минимален. Требуемое время работы $O(n^3 \log n)$}.\\
	\\
	Решение:\\
	Воспользуемся бинпоиском по весу ребра и алгоритмом Куна. Будем выбирать вес ребра и строим паросочетание, выбирая ребра, с весом меньшем или равным выбранным в бинпоиске. Если это паросочетание максимально, то сдвигаем правую границу, иначе левую. Таким образом за $O(n^3 \log n)$ найдем максимальное паросочетание, такое что вес максимального ребра минимален.\\
	\\
	\textbf{Задача 6.} \\
	\textit{Рёберным покрытием графа называется такое подмножество рёбер графа, что любая вер-
		шина является концом хотя бы одного из рёбер этого подмножества. Вам дан двудольный
		граф, требуется найти минимальное рёберное покрытие за время $O(nm)$.}\\
	\\
	Решение:\\
	Сначала построим максимальное паросочетание за время $O(nm)$(алгоритм Куна). \\
	Если паросочетание является совершенным паросочетанием, в котором все вершины уже покрыты, то нет необходимости в дополнительных рёбрах.\\ 
	Иначе пройдемся по всем ненасыщенным вершинам, каждая из ненасыщенных вершин смежна только с насыщенными вершинами, так как в противном случае паросочетание  можно было бы дополнить соответствующим ребром. И выбираем ребро, соединяющее насыщенную и ненасыщенную вершины. \\
	Таким образом получим минимальное рёберное покрытие за время $O(nm)$.
	
\end{document}
