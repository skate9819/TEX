\documentclass[12pt,a4paper]{scrartcl}
\usepackage[utf8]{inputenc}
\usepackage[english,russian]{babel}
\usepackage{indentfirst}
\usepackage{misccorr}
\usepackage{graphicx}
\usepackage{amsmath}

\begin{document}
	\begin{center}	
		Алгоритмы и структуры данных. Семинар 16.
		Деревья отрезков, LCA, разные задачи. \\
		Григорьев Дмитрий БПМИ-163\\
	\end{center}
	\textbf{Задача 4.} \\
	Заведем 3 указателя. Первый изначально в первой вернише, второй -- во второй.\\
	Первый указатель поднимем на 1, второй на два, затем первый еще на три (чтобы расстояние до вершины отправления стало 4), затем второй на 6 (чтобы расстояние до вершины отправления стало 8), затем первый на 12 (расстояние станет 16) и так далее, на i-ом шагу мы двигаем (i mod 2 + 1) - ой указатель так, чтобы его расстояние до вершины отправления стало равно $2^i$.
	Тогда указатели уже точно встретятся, когда они оба уже пересекут общего предка, и когда длина шага будет хотя бы 3 $d$, где $d$ -- разница в глубине (пусть длина шага равна 3 $q$, $q$ >= $d$. Тогда один из указателей находится на расстоянии 2 $q$ от своей вершины, а другой двигается от расстояния $q$ до расстояния 4 $q$ -- если он выше, то он встретит второй указатель будучи на расстоянии 2 $q$ - $d$ >= $q$, иначе на расстоянии 2 $q$ + $d$ < 4 $q$). Пройденное указателями расстояние пропорционально длине последнего совершенного шага, так что сложность O($\rho$). \\
	Когда указатели нашлись,считаем разницу растояний до вершин и поднимим нижнюю из них на один уровень. Дальше просто переходя к предкам найдем LCA. \\
	\\
	\\
	\textbf{Задача 8.} \\
	Эта структура данных -- дерево отрезков снизу.
	\\
	\\
	\\
	\textbf{Задача 9.} \\
	Изначально будем считать, что количество различных цветов в поддереве -- это размер поддерева. \\
	Далее запомним вершины в порядке DFS, храня для каждой вершины предыдущюю вершиину этого же цвета. \\
	Потом для текущей вершины и следующей такого же цвета, если таковая имеется, ищем LCA за ${O}(1)$ (алгоритм Фарах-Колтона и Бендера). И теперь обновляем количество различных цветов в поддереве: для LCA -- вычитаем 1. \\
	Теперь количество различных цветов в поддереве -- сумма в поддереве, с учетом обновлений. \\
	Время раюоты -- $O(n)$, так как LCA мы ищем за $O(1)$ и сумма в поддереве -- $O(n)$. 
	
\end{document}
