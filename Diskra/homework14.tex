\documentclass[12pt,a4paper]{scrartcl}
\usepackage[utf8]{inputenc}
\usepackage[english,russian]{babel}
\usepackage{indentfirst}
\usepackage{misccorr}
\usepackage{graphicx}
\usepackage{amsmath}

\begin{document}
	\begin{center}	
		Домашнне задание №14 \\
		Григорьев Дмитрий БПМИ--163
	\end{center}
	\textbf{Задание 1.} \\	
	\textbf{Решение:}
		Нам известно, что с 40\% от стоимости билетов идет на выигрыш. Значит с каждого билета в среднем на выигрыш уходит 40 рублей. Значит мат. ожидание выигрыша равно 40 рублей. \\
		Теперь рассмотрим неравенство Маркова: \\
		$Pr[f \geqslant a] \leqslant \frac{E[f]}{a}$ , где $f$ -- 	случайная величина, в нашем случае это выигрыш. \\
		Тогда $Pr[f \geqslant 5000] \leqslant \frac{40}{5000}$ \\
		$Pr[f \geqslant 5000] \leqslant \frac{1}{125} \leqslant 		\frac{1}{100}$\\
		Получилось, что вероятность выиграть 5000 меньше одного процента.\\
	\begin{flushright}	
		\textbf{ч.т.д.}
	\end{flushright} 
	\textbf{Задание 3.} 
	\\
	\textbf{Решение:}
	а) Расмотрим все возможные выигрыши первого и второго и сравним их средние значения. \\
	У первого среднее значение равно $\frac{(1 \cdot (1 + 2 + 3 + 4 + 5 + 6) + 2 \cdot (1 + 2 + 3 + 4 + 5 + 6) + .. + 6 \cdot (1 + 2 + 3 + 4 + 5 + 6))}{36} = 12,25$ \\
	У втрого среднее значение равно $\frac{1 + 4 + 9 + 16 + 25 + 36}{6} = 15\frac{1}{6}$ \\
	Получилось, что у второго средний выигрыш больше.\\
	б) Теперь будем считать, что вероятности выпадания кубика на грань соответственно равны $p1, p2, p3, p4, p5, p6$. Теперь распишем мат. ожидание для двух случаев. \\
	Для первого:
	$1\cdot(2p1p1) + 2\cdot(2p1p2) + 3\cdot(2p1p3) + 4\cdot(2p1p4 + 2p2p2) + 5\cdot(2p1p5) + 6\cdot(2p1p6 + 2p2p3) + 8\cdot(2p2p4) + 10\cdot(2p2p5) + 12\cdot(2p2p6 + 2p3p4) + 15\cdot(2p3p5) + 16\cdot(2p4p4) + 18\cdot(2p3p6) + 20\cdot(2p4p5) + 24\cdot(2p4p6) + 25\cdot(2p5p5) + 30\cdot(2p5p6) + 36\cdot(2p6p6)$ \\
	Для втрого: $1p1 + 4p2 + 9p3 + 16p4 + 25p5 + 36p6$\\
	Если из втрого вычесть первое, то полученная сумма будет больше или равна 0. \\
	Следовательно выигрыш втрого больше
	\begin{flushright}	
		\textbf{Ответ: Выигрыш больше у втрого}
	\end{flushright}
	\newpage
	\noindent
	\textbf{Задание 4.} 
	\\
	\textbf{Решение:} 
		Подслово $ab$ мы можем расположить на 19 местах, при этом нам нужно дописать 18 букв из $\{a, b\}$. Тогда получается $19 \cdot 2^{18}$ вариантов. \\
		Всего у нас $2^{20}$ варианов слов.
		Получается, что мат. ожидание равно $\frac{19 \cdot 2^{18}}{2^{20}} = 4,75$
	\begin{flushright}
		\textbf{Ответ: 4,75}
	\end{flushright}
	\textbf{Задание 5.} 
	\\
	\textbf{Решение:} 
	Для того, чтобы посчитать мат. ожидание количества завтраков, а для этого нужно знать вероятность того, что он попробует n завтраков.\\
	Из 10 вариантов выбрать n завтраков это $C_{10}^{n}$. Так же нам нужно учесть количество разбиений 15 дней на n завтраков. (Это неизвестные нам числа Стирлинга второго рода). Всего у нас есть $10^{15}$ вариантов "последовательностей завтраков". Получается, вероятность того, что проректор попробует n завтраков равна $\frac{C_{10}^{n}\cdot S(15,k)}{10^{15}}$.\\
	Теперь нужно посчитать мат.ожидание по формуле $\sum\limits_{i=1}^{10} i\cdot\frac{C_{10}^{i}\cdot S(15,i)}{10^{15}}$. Если это посчитать то это будет примерно 7,9
	\begin{flushright}
		\textbf{Ответ: 7,9}
	\end{flushright}
	\textbf{Задание 6.} 
	\\
	\textbf{Решение:} 
		Рассмотрим какую то перестановку. Для нее найдется "перевернутая" перестановка. В сумме у них будет $\frac{n\cdot(n - 1)}{2}$ инверсий.\\
		Тогда у всех перестановок длины n в сумме будет $\frac{n\cdot(n - 1)}{2} \cdot \frac{n!}{2}$.\\
		Теперь найдем мат.ожидание количества инверсий, разделив общее количество инверсий на количество всех перестановок. 
		$E[I(\pi)] = \frac{\frac{n\cdot(n - 1)}{2} \cdot \frac{n!}{2}} {n!} = \frac{n \cdot(n - 1)}{4}$
	\begin{flushright}
		\textbf{Ответ:$\frac{n \cdot(n - 1)}{4}$}
	\end{flushright}
	\textbf{Задание 7.} 
	\\
	\textbf{Решение:}
	Нам необходимо доказать, что $Pr[X \geqslant 6] < \frac{1}{10}$. \\
	Заменим это утверждение на $Pr[2^X \geqslant 64] < \frac{1}{10}$\\
	Тогда по неравенству Маркова: $Pr[2^X \geqslant 64] < \frac{E[X]}{64} => Pr[2^X \geqslant 64] < \frac{5}{64}$. \\
	Получилось, что $Pr[X \geqslant 6] < \frac{5}{64} => Pr[X \geqslant 6] < \frac{1}{10}$.
	\begin{flushright}
		\textbf{ч.т.д.}
	\end{flushright}
\end{document}
