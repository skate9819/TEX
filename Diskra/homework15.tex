\documentclass[12pt,a4paper]{scrartcl}
\usepackage[utf8]{inputenc}
\usepackage[english,russian]{babel}
\usepackage{indentfirst}
\usepackage{misccorr}
\usepackage{graphicx}
\usepackage{amsmath}

\begin{document}
	\begin{center}	
		Домашнне задание №15 \\
		Григорьев Дмитрий БПМИ--163
	\end{center}
	\textbf{Задание 1.} \\	
	\textbf{Решение:}
	\\Заметим, что A $\backslash$ B = A $\backslash$ (A $\cap$ B). Так как A $\backslash$ B бесконечно, то A $\backslash$ (A $\cap$ B) тоже бесконечно, а это значит мы можем выделить счетное подмножество C в A $\backslash$ (A $\cap$ B). Теперь заметим, что 
	C $\cup$ (A $\cap$ B) - счетно, так как C счетно и (A $\cap$ B) либо счетно, либо конечно. \\
	Пусть у нас есть биекция из  (A $\backslash$ (A $\cap$ B)) $\backslash$ С в себя, биекция из С в C $\cup$ (A $\cap$ B) (так как и то и то счетно). \\
	Получилось, что из A $\backslash$ (A $\cap$ B) установилось биективное соответствие в A, так как у нас есть биекция для A $\backslash$ (A $\cap$ B)) $\backslash$ С и для С. \\
	А в начале мы в начале сказали, что A $\backslash$ B = A $\backslash$ (A $\cap$ B), следовательно у нас есть биекция из A $\backslash$ B в А, значит они равномощны.
	\begin{flushright}	
		\textbf{Ответ: верно}
	\end{flushright} 
	\textbf{Задание 2.} 
	\\
	\textbf{Решение:}
		\\Если A = B $\neq \varnothing$, то A $\bigtriangleup$ B = $\varnothing$. Получается, что A $\bigtriangleup$ B не равномощно A, так как пустое множество не может быть равномощным с непустым.
	\begin{flushright}	
		\textbf{Ответ: не верно}
	\end{flushright}
	\noindent
	\textbf{Задание 3.} 
	\\
	\textbf{Решение:} 
	\\Заметим, что A $\backslash$ B = A $\backslash$ (A $\cap$ B). Множество A $\backslash$ (A $\cap$ B) бесконечно, так как A $\cap$ B - конечно, а это значит мы можем выделить счетное подмножество C в A $\backslash$ (A $\cap$ B). Теперь заметим, что 
	C $\cup$ (A $\cap$ B) - счетно, так как C счетно и (A $\cap$ B) конечно. \\
	Пусть у нас есть биекция из  (A $\backslash$ (A $\cap$ B)) $\backslash$ С в себя, биекция из С в C $\cup$ (A $\cap$ B) (так как и то и то счетно). \\
	Получилось, что из A $\backslash$ (A $\cap$ B) установилось биективное соответствие в A, так как у нас есть биекция для A $\backslash$ (A $\cap$ B)) $\backslash$ С и для С. \\
	А в начале мы в начале сказали, что A $\backslash$ B = A $\backslash$ (A $\cap$ B), следовательно у нас есть биекция из A $\backslash$ B в А, значит они равномощны.
	\begin{flushright}
		\textbf{Ответ: верно}
	\end{flushright}
	\newpage
	\noindent
	\textbf{Задание 4.} 
	\\
	\textbf{Решение:} 
	\\ Пусть у нас имеется какое--то множество непересекающихся интервалов. Если их конечное число, то задача решена.\\
	Иначе, мы можем на каждом интервале выбрать рациональное число, так как интервалы непересекаются, и у нас полилось счетное множество интервалов, так как множество рациональных чисел счетно.
	\begin{flushright}
	\textbf{ч.т.д.}
	\end{flushright}
	\textbf{Задание 5.} 
	\\
	\textbf{Решение:} 
	\\ Из бесконечного множества можно выделить счетное подмножество и разбить его на счетное подмножество счетных подмножеств, это сделать не сложно:
	\\ упорядочим все пары $(i, j),$ такие, что i, j $\epsilon$ $ \mathbb{N}$. Теперь сопостаавим их элементам выделеного счетного подмножества. Разобьем это счетное подмножество на счетное подмножество счетных подмножеств, где каждому элементу соответствует пара индексов (i, j), с фиксированным j и любым натуральным i.
	\\Получилось, что бесконечное множество можно разбить на счетное колиество счетных подмножеств.
	\begin{flushright}
		\textbf{ч.т.д.}
	\end{flushright}
	\textbf{Задание 7.} 
	\\
	\textbf{Решение:}
	\\ Пусть множество всех строго возрастающих последовательностей - это А, а множество последовательностей натуральных чисел - это В. Мы знаем, что $\forall i: a_i < a_{i + 1}$. Мы можем единственным образом задать $B_i$ через последовательность из А:\\
	\{ $b_1 = a_1$,  $b_{i + 1} = a_{i + 1} - a_i $ \}. \\
	Таким образом, у нас есть явная биекция между A и B. 
	\begin{flushright}
		\textbf{ч.т.д.}
	\end{flushright}
\end{document}
