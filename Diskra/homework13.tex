\documentclass[12pt,a4paper]{scrartcl}
\usepackage[utf8]{inputenc}
\usepackage[english,russian]{babel}
\usepackage{indentfirst}
\usepackage{misccorr}
\usepackage{graphicx}
\usepackage{amsmath}

\begin{document}
	\begin{center}	
		Домашнне задание №13 \\
		Григорьев Дмитрий БПМИ--163
	\end{center}
	\textbf{Задание 1.} \\
	\textbf{Решение:}
		Рассмотрим возможные исходы:\\
		1) Сначала родился мальчик, потом девочка\\
		2) Сначала родился мальчик, потом еще один мальчик\\
		3) Сначала девочка, потом мальчик\\
		Как видно вариантов которые нам подходят -- 2, значит искомая вероятность --
		$\frac{2}{3}$.\\
		Так же можно посчитать через формулу условных вероятностей: $Pr[A|B] = \frac{Pr[A \cap B]}{Pr[B]}$. Где A -- один ребенок в семье девочка, B - один ребенок мальчик. \\
		$Pr[A \cap B] = 0.5$, $Pr[B] = \frac{2}{3}$.\\
		$Pr[A|B] = \frac{2}{3}$
	\begin{flushright}	
		\textbf{Ответ:} $\frac{2}{3}$
	\end{flushright} 
	\textbf{Задание 2.}\\
	\textbf{Решение:}
		Множество исходов в данной задаче  -- числа от 1 до n. Вероятность каждого события одинакова и равна $\frac{1}{n}$\\
		Количество исходов, подходящих под условие того, что выбранное число делится на 2 равна $\lfloor \frac{n}{2} \rfloor$ \\
		Значит вероятность того, что выбранное число делится на 2 равна $\frac{\lfloor \frac{n}{2} \rfloor}{n}$. \\
		Аналогично рассуждая, получим, что вероятность того, что выбранное число делится на 3 равна $\frac{\lfloor \frac{n}{3} \rfloor}{n}$.\\
		События будут независимы, кагда $P[B|A] = P[B]$, т.е.
		$P[B \cap A] = P[B] \cdot P[A]$.\\
		$P[B \cap A] = \frac{\lfloor \frac{n}{6} \rfloor}{n}$ и это должно быть равно $\frac{\lfloor \frac{n}{2} \rfloor}{n} \cdot \frac{\lfloor \frac{n}{3} \rfloor}{n}$.
		Рассмотрим число n, с точки зрения деления на 6:\\
		n = 6m, тогда проверим наше равенство $n \cdot m = 3m \cdot 2m$. Оно верно! \\
		n = 6m + 1, тогда проверим наше равенство $(6m + 1) \cdot m = 6m^2$. Оно верно при m = 0, значит n = 1! \\
		n = 6m + 2, тогда проверим наше равенство $(6m + 2) \cdot m = (3m + 1) \cdot 2m$. Оно верно! \\
		n = 6m + 3, тогда проверим наше равенство $(6m + 3) \cdot m = (3m + 1) \cdot (2m + 1)$. Оно неверно! \\
		n = 6m + 4, тогда проверим наше равенство $(6m + 4) \cdot m = (3m + 2) \cdot (2m + 1)$. Оно неверно! \\
		n = 6m + 5, тогда проверим наше равенство $(6m + 5) \cdot m = (3m + 2) \cdot (2m + 1)$. Оно неверно! \\
	\begin{flushright}
		\textbf{Ответ:} при n = 1, n = 0 (mod 6), n = 2 (mod 6)

	\end{flushright}
	\textbf{Задание 3.} 
	\\
	\textbf{Решение:} События будут независимы, кагда $P[B|A] = P[B]$, т. е. \\ 
	$P[B \cap A] = P[B] \cdot P[A]$.\\
	$P[B \cap A] = \frac{C_{36}^{3}}{C_{36}^{5}} = \frac{7140}{376992}$\\ 
	$P[B] = P[A] = \frac{C_{36}^{4}}{C_{36}^{5}} = \frac{58905}{376992}$\\
	Теперь проверим являются ли события независимыми: \\
	$\frac{58905}{376992} \cdot \frac{58905}{376992} != \frac{7140}{376992}$ \\
	Получается, что события не являются независимыми.
	\begin{flushright}	
		\textbf{Ответ:} события не являются независимыми.
	\end{flushright}
	\noindent
	\textbf{Задание 4.} 
	\\
	\textbf{Решение:} Множество вероятных исходов - это множество всех функций, а их всего -- $n^n$.\\
	Иньективных функций будет $n!$, следовательно вероятность того, что выбранная функция инъективна это $\frac{n!}{n^n}$. \\
	Теперь посмотрим, что будет, когда $f(1) = 1$. Вероятость того, что функция инъективна при условии $f(1) = 1$ равна $\frac{(n - 1)!}{n^{n - 1}}$. \\
	Заметим, что $\frac{(n - 1)!}{n^{n - 1}} = \frac{(n)!}{n^{n}}$.\\
	Следовательно события независимы.
	\begin{flushright}
		\textbf{Ответ:} события являются независимыми 
	\end{flushright}
	\textbf{Задание 5.} 
	\\
	\textbf{Решение:} Множество вероятных исходов - принятые решения членов жюри. \\
	Вероятность того, что два члена жюри примут верное решение это $p \cdot p$. Если же один из членов жюри примет верное решение, а другой нет, то вероятность такого равна $p \cdot (1 - p) \cdot 2$. В этой ситуации все зависит от от третьего члена жюри. Он примет правильно решение с вероятностью $\frac{1}{2} \cdot (2p - 2p^2) = p - p^2$\\
	Теперь посчитаем вероятность того, что будет принято верное решение -- \\ $p - p^2 + p^2 = p$.
	
	\begin{flushright}
		\textbf{Ответ:} p
	\end{flushright}
	\textbf{Задание 6.} 
	\\
	\textbf{Решение:} Предположим, что мы знаем что в одном ящиек лежит a черных и b белых шаров, следовательно в другом лежит (10 - a) черных и (10 - b) белых. Тогда вероятность того, что узника не казнят равна $\frac{\frac{b}{a + b} + \frac{10 - b}{20 - a - b}}{2}$. Нам необходимо посчитать, когда эта вероятность будет максимальной.\\
	С помощью программы это легко сделать. В результате получается, что наибольшая вероятность будет при a = 0 и b = 1, и она равна $\frac{14}{19}$ (можно было перебрать руками).
	\begin{flushright}
		\textbf{Ответ:} Нужно в один ящик положить только 1 белый шар и в другой 10 черных и 9 белых. Тогда $P = \frac{14}{19}$
	\end{flushright}
	\textbf{Задание 7.} 
	\\
	\textbf{Решение:} Нужно найти вероятность победы первого при текущем счете 8:7. Всего остается сыграть 5 партий, значит можно перебрать всевозможные исходы 4 партий, потому что после 4 партий будет понятно кто выиграл. \\ 
	Пусть 1 - выиграл первый, 0 - первый проиграл. Тогда переберем всевозможные последовательности из 0 и 1 длины 4. Чтобы первый выиграл матч, ему необходимо выиграть хотя бы 2 партии, значит нужно посчитать кол-во последовательностей длины 4, в которых не меньше двух 1. \\
	Это последовательности: 0000, 0001, 0010, 0100, 1000\\
	Всего последовательностей длины 4 :  $2^4 = 16$, а не подходит нам 5 из них.\\
	Следовательно в (16 - 5 = 11) случаях в матче побеждает первый.
	Искомая вероятность -- $\frac{11}{16}$.
		\begin{flushright}
		\textbf{Ответ:} $\frac{11}{16}$
	\end{flushright}
\end{document}
