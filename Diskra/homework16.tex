\documentclass[12pt,a4paper]{scrartcl}
\usepackage[utf8]{inputenc}
\usepackage[english,russian]{babel}
\usepackage{indentfirst}
\usepackage{misccorr}
\usepackage{graphicx}
\usepackage{amsmath}

\begin{document}
	\begin{center}	
		Домашнне задание №16 \\
		Григорьев Дмитрий БПМИ--163
	\end{center}
	\textbf{Задание 1.}
	\newline
	\textbf{Решение:}
	\newline
	\indent
	У каждого круга на плоскости мы знаем координаты его центра и длину радиуса. Поэтому для каждого круга мы можем сопоставить тройку чисел: координаты центра и радиус. Для всех кругов у нас получается получается биекция $\mathbb{R}^2$$\times$$\acute{\mathbb{R}}$, где $\acute{\mathbb{R}}$ - это множество положительных вещественных чисел. Получается декартово произведение трех континуальных множеств, что тоже является конинуумом. Значит множество всех кругов на плоскости это континуум.
	\begin{flushright}	
		\textbf{Ответ: верно}
	\end{flushright} 
	\textbf{Задание 2.} 
	\newline
	\textbf{Решение:}
	\newline
	\indent
	Не верно, в качестве примера возьмем множество концентрических окружностей. Тогда все окружности задаются только своим радиусом, так как центр это только одна точка, поэтому множество всех таких окружностей не континуум. 
	
	\begin{flushright}	
		\textbf{Ответ: не верно}
	\end{flushright}
	\noindent
	\textbf{Задание 3.} 
	\newline
	\textbf{Решение:} 
	\newline
	\indent
	Мы знаем, что плоскость имеет мощность конинуума, рассмотрим прямые на плоскости вида $y = a$, где $a$ -- констаната, $a \in \mathbb{R}$. Видно, что множество всех констант континуально, значит и количество прямых тоже континуум. Сама прямая это тоже континуум. Поэтому семейство всех таких прямых и есть континуальное семейство попарно непересекающихся континуальных подмножеств в $\mathbb{R}$.  
	\begin{flushright}
		\textbf{Ответ: да, существует}
	\end{flushright}
	\noindent
	\textbf{Задание 4.} 
	\newline
	\textbf{Решение:} 
	\newline 
	Рассмоирим множество таких последовательностей. Заметим, что у нас более двух единиц подряд не могут стоять рядом. Заменим две единицы, стоящие рядом на '2'. Так же заметим, что теперь после '1' или '2' может стоять только '0'. Тогда заменим все '10' и '20' на '1' и '2' соответственно. Тогда у нас получается множество последовательностей, состоящих из '0', '1', '2'. А мы знаем, что такое множество континуум.
	\begin{flushright}
	\textbf{Ответ: верно}
	\end{flushright}
	\newpage
	\noindent
	\textbf{Задание 6.}
	\newline
	\textbf{Решение:}
	\newline
	\indent
	У нас есть континуум непересекающихся равных единиц. Мы можем расположить единицу так, что ее угол находится в начале координат, а один из отрезков лежит на оси OX. Тогда выделем прямую, которая проходит через начало координат и лежит внутри угла единицы. И теперь просто передвигаем единицу вдоль этой прямой. Таким образом мы расположим единички так, что они не будут пересекаться. Так как прямая это континуум, то мы расположим на ней все единички. 
	\begin{flushright}
		\textbf{Ответ: да}
	\end{flushright}
	\textbf{Задание 7.} 
	\newline
	\textbf{Решение:}
	\newline
	\indent
	У нас должны быть непересекающиеся восьмерки, значит их внутренности тоже не пересекаются. Внутри любой окружности мы можем найти точку с рациональными координатами(так как между двумя рациональными числами можно выбрать другое рациональное число). Так как восьмерка содержит две окружности, то любая такая окружность будет задаваться четырьмя рациональными числами. Значит множество непересекающихся восьмереок иньективно $\mathbb{Q}^4$, а это значит, что множество непересекающихся восьмереок не более чем счетно(так как $\mathbb{Q}^4$ счетно).
	\begin{flushright}
		\textbf{Ответ: нельзя}
	\end{flushright}
\end{document}
