\documentclass[12pt,a4paper]{scrartcl}
\usepackage[utf8]{inputenc}
\usepackage[english,russian]{babel}
\usepackage{indentfirst}
\usepackage{misccorr}
\usepackage{graphicx}
\usepackage{amsmath}

\begin{document}
	\begin{center}	
		Домашнне задание №23 \\
		Григорьев Дмитрий БПМИ--163
	\end{center}
	\textbf{Задание 1.}
	\newline
	\textbf{Решение:}
	\newline
	\indent
	Так как нам необходима МТ, которая вычисляет нигде не определенную функцию, то нужно построить такую МТ, которая не останавливается при любых входных данных. Построим таблицу переходов, где 0 -- начальное состояние:\\
	$\delta :
	\begin{cases}
	(a, 0) \longmapsto (a, 0, 0)\\
	(\varLambda, 0) \longmapsto (\varLambda, 0, 0)\\	
	\end{cases}
	$ \\
	Где $a$ -- какой--либо символ из алфавита, а $\varLambda$ -- пустой символ. Тогда головка этой МТ будет постоянно переходить в тот символ, где она была, т. е. бесконечно переходить в символ, на котором она стоит.
	\begin{flushright}	
		\textbf{}
	\end{flushright}
	\textbf{Задание 2.} 
	\newline
	\textbf{Решение:} 
	\newline
	\indent
	Построим нужную МТ, такую, что 0 -- начальное состояние: \\
	$\delta :
	\begin{cases}
	(0, 0) \longmapsto (1, 0, +1)\\
	(1, 0) \longmapsto (0, 0, +1)\\
	(\varLambda, 0) \longmapsto (\varLambda, 1, -1)\\
	(0, 1) \longmapsto (0, 1, -1)\\
	(1, 1) \longmapsto (1, 1, -1)\\
	(\varLambda, 1) \longmapsto (\varLambda, 2, +1)
	\end{cases}
	$ \\
	Эта МТ переводит конфигурацию 0$w$ в $\overline{w}$0. Далее переходит к 1 состоянию. На 1 состоянии МТ переходит к началу слова, не меняя его, т. е. к конфигурации $1\varLambda\overline{w}$. Далее один раз выполняется последняя строка таблицы и во 2 состоянии завершается работа МТ, при этом результат -- $\overline{w}$.
	\begin{flushright}
		\textbf{}
	\end{flushright}
	\textbf{Задание 4.}
	\newline
	\textbf{Решение:}
	\newline
	\indent
	Рассмотрим следующую МТ. Нам нужно сдвинуть все $1$ вправо, поэтому воспользуемся алгоритмом: \\
	1)Найдем первую единицу справа от головки(если ее нет, то просто завершаем работу)\\
	2)Поставим вместо этой единицы символ $\varLambda$\\
	3)Найдем первый 0 справа от головки(если его нет, то поменяем символ $\varLambda$ назад на 1 и вернемся в начало и завершим работу)\\
	4)Поменяем этот ноль на 1\\
	5)Пустой символ $\varLambda$ заменим на 0 и вернемся к пункту 1)\\
	Таким образом мы получим нужное слово $0^a1^b$, следовательно существует такая МТ.
	\begin{flushright}
		\textbf{ч.т.д.}
	\end{flushright}
	
	\noindent
	\textbf{Задание 3.} 
	\newline
	\textbf{Решение:}
	\newline
	\indent
	Нам нужно узнать, существует ли в данном слове подслово $aba$, зная что алфавит -- $\{a, b, c\}$. Для этого просто будем идти по слову и проверять, есть ли это подслово, 0 -- начальное состояние:\\
	$
	\delta: 
	\begin{cases}
	(a, 0) \longmapsto (\varLambda, 1, +1) \\
	(b, 0) \longmapsto (\varLambda, 0, +1) \\	
	(c, 0) \longmapsto (\varLambda, 0, +1) \\
	(a, 1) \longmapsto (\varLambda, 0, +1) \\
	(b, 1) \longmapsto (\varLambda, 2, +1) \\
	(c, 1) \longmapsto (\varLambda, 0, +1) \\
	(a, 2) \longmapsto (\varLambda, 3, +1) \\
	(b, 2) \longmapsto (\varLambda, 0, +1) \\
	(c, 2) \longmapsto (\varLambda, 0, +1) \\
	(a, 3) \longmapsto (\varLambda, 3, +1) \\
	(b, 3) \longmapsto (\varLambda, 3, +1) \\
	(c, 3) \longmapsto (\varLambda, 3, +1) \\
	(\varLambda, 0) \longmapsto (0, 4, 0) \\
	(\varLambda, 1) \longmapsto (0, 4, 0) \\
	(\varLambda, 2) \longmapsto (0, 4, 0) \\
	(\varLambda, 3) \longmapsto (1, 4, 0) \\
	\end{cases}
	$ \\
	В состоянии 0 находимся до тех пор, пока не найдем символ $a$, далее переходим в состояние 1. Если далее идет символ $b$, то переходим в состояние 2, иначе опять возвращаемся в 0 состояние. Далее, мы можем находиться в состоянии 2 только если сейчас уже есть подслово $ab$, следовательно если следующий символ -- $a$, то переходим в состояние 3(которое означает, что существует подслово $aba$), иначе опять в состояние 0. Когда мы встречаем $varLambda$ в 3 состоянии, то выводим 1, иначе 0. Таким образом мы узнаем -- есть ли подслово $aba$.
	\begin{flushright}	
		\textbf{}
	\end{flushright}
	\textbf{Задание 5.} 
	\newline
	\textbf{Решение:}
	\newline
	\indent
	Рассмотрим такую МТ. Она будет содержать 2 ленты. Для начала скопируем слово на вторую ленту. Отавим головку на второй ленте на конце. Далее идем по слову на первой ленте вправо, а по второй -- влево. Если у нас попались несовпадаемые символы -- значит это не палиндром и в этой ситуации удаляем оба слова и ставим 0 на 1 ленте. Если же мы прошли полностью слово, то получается, что оно палиндром -- значит ставим 1.
	\begin{flushright}	
	\textbf{}
	\end{flushright}
	\textbf{Задание 6.} 
	\newline
	\textbf{Решение:}
	\newline
	\indent
	Да, существует. Рассмотрим такую МТ, у которой 51 лента, причем на первой ленте напишем 17 единиц, а на 50 остальных по 40. Далее будем ходить по всем единицам, начиная со второй ленты. При каждом попадании на 1 дописываем на первую ленту единицу и пререходим к следующей -- если следующей нет, то переходим на следующую ленту. Получится, что у нас на 1 ленте будет ровно 2017 единиц и состояний будет меньше 100.
	\begin{flushright}	
		\textbf{Ответ: существует}
	\end{flushright}
	\textbf{Задание 7.} 
	\newline
	\textbf{Решение:}
	\newline
	\indent
	Выберем такую вычислимую биекцию. Так как она вычислима -- следовательно мы можем посчитать ее на МТ. Тогда посчитаем $f(a, b)$ и выпишем на ленту $1^{f(a, b)}$. 
	\begin{flushright}	
		\textbf{ч.т.д.}
	\end{flushright}
\end{document}
