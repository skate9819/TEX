\documentclass[12pt,a4paper]{scrartcl}
\usepackage[utf8]{inputenc}
\usepackage[english,russian]{babel}
\usepackage{indentfirst}
\usepackage{misccorr}
\usepackage{graphicx}
\usepackage{amsmath}

\begin{document}
	\begin{center}	
		Домашнне задание №22 \\
		Григорьев Дмитрий БПМИ--163
	\end{center}
	\textbf{Задание 1.}
	\newline
	\textbf{Решение:}
	\newline
	\indent
	Множество программ, которым на вход подается один аргумент $x$ и выводится число $2017$, счетно. Каждая такая программа имеет свой номер $p$ и для него выполняется $U(p, x) = 2017$.
	\begin{flushright}	
		\textbf{ч.т.д.}
	\end{flushright}
	\textbf{Задание 3.} 
	\newline
	\textbf{Решение:} 
	\newline
	\indent
	Известно, что $U(p, x)$ -- главная универсальная вычислимая функция, следовательно существует транслятор $s(q)$ —
	вычислимая всюду определённая функция, для которой выполняется 
	$V(q, x) = U(s(q), x)$. $^{*}$\\
	Так же по теореме о неподвижной точке для любой всюду определённой вычислимой функции $p(t)$ существует такое
	$t$, что $U(p(t), x) = U(t, x)$.  $^{**}$\\
	Тогда из $^{*}$ и $^{**}$ следует, что $V(p, x) = U(s(p), x) = U(p, x)$. Следовательно найдется такое $p$, что $V(p, x) = U(p, x)$ для всех $x$.
	\begin{flushright}
		\textbf{ч.т.д.}
	\end{flushright}
	\noindent
	\textbf{Задание 2.} 
	\newline
	\textbf{Решение:}
	\newline
	\indent
	Мы можем воспользоваться Заданием 3. Если мы возьмем, что $V(p, x) = px$, тогда, воспользовавшись доказанным Заданием 3, получим, что найдется такое $n$, что $U(n, x) = nx$ для всех $x$.
	\begin{flushright}	
		\textbf{ч.т.д.}
	\end{flushright}
	\textbf{Задание 4.}
	\newline
	\textbf{Решение:}
	\newline
	\indent
	Рассмотрим множество всех вычислимых функций, определенных в $0$ -- $A$. $A$ это нетривиальные свойство вычислимых функций. \\
	$U$ -- главная универсальная вычислимая функция, тогда по теореме
	Успенского--Райса множество $I = {p : U(p, x) \in A}$ неразрешимо. Если же $I$ -- это множество четных чисел, то такое множество разрешимо (алгоритм
	разрешения множества просто проверяет число на четность). Получили противоречие. Следовательно, такого $U$ не существует.
	\begin{flushright}
		\textbf{Ответ: нет, не существует}
	\end{flushright}
	
	\noindent

\end{document}
