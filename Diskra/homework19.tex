\documentclass[12pt,a4paper]{scrartcl}
\usepackage[utf8]{inputenc}
\usepackage[english,russian]{babel}
\usepackage{indentfirst}
\usepackage{misccorr}
\usepackage{graphicx}
\usepackage{amsmath}

\begin{document}
	\begin{center}	
		Домашнне задание №19 \\
		Григорьев Дмитрий БПМИ--163
	\end{center}
	\textbf{Задание 1.}
	\newline
	\textbf{Решение:}
	\newline
	\indent
	Нам задан граф из $\binom{n}{2}$ 0 или 1. Он задан в виде верхнего правого треугольника матрицы смежности. $x_{i, j}$ равен одному, если $i$-ая и $j$-ая вершины соедины и равен нулю, если не соединены. Вершина $i$ будет изолированной, если в матрице смежности $i$-ый столбец или $i$-ая строка нулевые. Нужно просто проверить, есть ли в заданном "треугольнике" такие стлбцы или строки: \\
	$\bigvee_{i = 1}^{n} \neg((\bigvee_{j = 1}^{i - 1} x_{i j}) \lor (\bigvee_{j = i + 1}^{n} x_{i j}))$. \\
	Таким образом, мы посчитаем есть ли в графе изолированые выршины.
	\begin{flushright}	
		\textbf{}
	\end{flushright}
	\textbf{Задание 2.} 
	\newline
	\textbf{Решение:}
	\newline
	\indent
	Нам задан граф из $\binom{n}{2}$ 0 или 1. Он задан в виде верхнего правого треугольника матрицы смежности. $x_{i, j}$ равен одному, если $i$-ая и $j$-ая вершины соедины и равен нулю, если не соединены.
	Три вершины $i, j, k$ будут образовывать треугольник, если $x_{ij} = x_{jk} = x_{ki} = 1$. Нужно просто проверить, нет ли в заданном графе три такие вершины: \\
	$\neg(\bigvee_{i = 1}^{n - 2} \bigvee_{j = i + 1}^{n - 1} \bigvee_{k = j + 1}^{n} (x_{ij} \land x_{jk} \land x_{ik}))$ \\
	Таким образом, мы проверим -- есть ли в графе треугольник.
	
	\begin{flushright}	
		\textbf{}
	\end{flushright}
	\textbf{Задание 3.} 
	\newline
	\textbf{Решение:} 
	\newline
	\indent
	Нам задан граф из $\binom{n}{2}$ 0 или 1. Он задан в виде верхнего правого треугольника матрицы смежности. $x_{ij}$ равен одному, если $i$-ая и $j$-ая вершины соедины и равен нулю, если не соединены. Граф будет содержать эйлеров цикл, если он связен и степень каждой вершины будет четной. На связность граф можно проверить, возведением матрицы смежности в степень -- это было на лекции. Проверить вершину на четность степени можно с помощью xor'а, так как это и есть сумма по модулю 2. Это нужно проверить для каждой вершины: \\
	$\bigwedge_{i = 1}^{n} \neg((\bigoplus_{j = 1}^{i - 1} x_{ji}) \oplus (\bigoplus_{j = i + 1}^{n} x_{ij}))$\\
	Таким образом, мы проверим -- четны ли степени всех вершин. Теперь, зная связен ли граф и четны ли степени всех вершин, мы можем ответить на вопрос.	
	\begin{flushright}
		\textbf{}
	\end{flushright}
	\noindent
	\textbf{Задание 4.} 
	\newline
	\textbf{Решение:} 
	\newline
	\indent
	Любую монотонную функцию модно представить в виде ДНФ без отрицаний. Максимум в такой форме может быть $2^n$ операций дизъюнкции, а в каждой конъюнкции может быть $n$ элментов. Получается что размер функции будет $O(n \cdot 2^n)$.
	\begin{flushright}
	\textbf{ч.т.д.}
	\end{flushright}
	\noindent
	\textbf{Задание 6.}
	\newline
	\textbf{Решение:}
	\newline
	\indent
	Любую функцию можно представить в виде полинома Жегалкина в базисе $\{\oplus, \cdot, 1 \}$, т.е. в виде:\\
	$P(x_1, x_2, ... , x_n) = a \oplus a_1 \cdot x_1 \oplus ... \oplus a_n \cdot x_n \oplus a_{1,2} \cdot x_1 \cdot x_2 \oplus a_{1,3} \cdot x_1 \cdot x_3 \oplus ... \oplus a_{1...n} \cdot x_1 \cdot ... \cdot x_n$ \\
	Как видим, всего будет $2^n$ слагаемых. Каждое слагаемое можно представить в виде двоичного числа, у которого в $i$-ой позиции стоит 1, если в слагаемом есть $x_i$ и 0 иначе. Xor всех слагаемых мы можем вычислить схемой длины $2^n - 1$. Сами слагаемые мы можем вычислять за $O(1)$ через предыдущие. Так как у нас $2^n$ слагаемых, то и размер схемы для них -- $2^n$.\\
	И всего получается $2 \cdot 2^n - 1$, что меньше $2^{n + 1}.$
	\begin{flushright}
		\textbf{ч.т.д.}
	\end{flushright}
	\textbf{Задание 7.}
	\newline
	\textbf{Решение:}
	\newline
	\indent
	Докажем это по индукции.\\
	База: Функция, в схема которой состоит из 1 функци\\
	Шаг индукции: пусть первые $i$ функций в схеме линейны. Тогда $i + 1$-ая функция вычисляется через предыдущие, которые уже линейны: \\
	$f_{i + 1} = k_1 \cdot f_1 \oplus ... \oplus k_i \cdot f_i = k_1 \cdot (a_{1, 0} \oplus a_{1, 1} \cdot x_1 \oplus ... \oplus a_{1, n} \cdot x_n) \oplus ... \oplus k_i \cdot (a_{i, 0} \oplus a_{i, 1} \cdot x_1 \oplus ... \oplus a_{i, n} \cdot x_n) = b_0 \oplus (b_1 \land x_1) \oplus ... \oplus (b_n \land x_n)$. \\
	Получилось, что $f_{i + 1}$ функция тоже линейна. По индукции получается, что функция, в схеме которой только линейные функции, тоже линейна. 
	\begin{flushright}
		\textbf{ч.т.д.}
	\end{flushright}

\end{document}
