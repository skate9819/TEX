\documentclass[12pt,a4paper]{scrartcl}
\usepackage[utf8]{inputenc}
\usepackage[english,russian]{babel}
\usepackage{indentfirst}
\usepackage{misccorr}
\usepackage{graphicx}
\usepackage{amsmath}

\begin{document}
	\begin{center}	
		Домашнне задание №12 \\
		Григорьев Дмитрий БПМИ--163
	\end{center}
	\textbf{Задание 1.} \\
	\textbf{Решение:}
	Пусть нам дано бинарное отношение P, определенное на множестве A. Тогда количество всевозможных бинарных отношений на A -- это $2^{|A|\cdot|A|}$, т.е. -- 16. \\
	Вместо того, чтобы рассматривать транзитивные отношения рассмотрим нетранзитивные отношения. Оно нетразитивно, если (x, y) и (y, x) содержатся в P, а (x, x) или (y, y) не содержатся в P. Таких множеств только 3, значит транзитивных -- \\ 16 - 3 = 13.
	\\
	Получается, что вероятность того, что бинарное отношение тразитивно равна $\frac{13}{16}$.
	\begin{flushright}	
		\textbf{Ответ:} { $\frac{13}{16}$}.
	\end{flushright} 
	\textbf{Задание 2.}\\
	\textbf{Решение:}
	Заметим, что при $a \neq b$ нельзя провести биекцию, значит вероятность того, что $f$ -- биекция, равна $0$. \\
	Теперь рассмотрим случай, когда $a = b$. \\
	Всевозможных функций у нас $a^a$, т. к. каждому элементу из множества $A$ сопостовляется не больше одного элемента из $B$. \\
	Теперь посчитаем количество биективных функций. Так как для каждого элемента из A мы выбираем один элемент из B, так что для каждому элементу из B соответствует не больше $1$ элемента из $A$. Следовательно всего биективных функций -- $a!$ Вероятность того, что $f$ -- биекция -- $\frac{a!}{a^a}$.\\
	\begin{flushright}
		\textbf{Ответ:} Если $a \neq b$, то $0$, иначе $\frac{a!}{a^a}$.
	\end{flushright}
	\textbf{Задание 3.} 
	\\
	\textbf{Решение:}
	Количество всевозможных перестановок чисел от $1$ до $37$ -- это $37!$ 
	\\
	Теперь рассмотрим $2$ ситуации: \\
	1) если $37$ стоит среди первых $18$ чисел -- таких перестановок $18 \cdot 36!$ 	
	\\
	2) если $37$ стоит на $19$ месте, тогда $36$ должно стоять среди первых $18$ чисел -- таких перестановок $18 \cdot 35!$ 
	\\
	Получается, что количество перестановок, когда наибольшее число среди первых $18$ чисел больше наибольшего числа среди последних $18$ чисел это -- $18 \cdot 36! + 18 \cdot 35!$ 
	\\		
	Искомая вероятность -- 	
	$\frac{18 \cdot 36! + 18 \cdot 35!}{37!}$ = $\frac{18 \cdot 35! \cdot (1 + 36)}{37!}$ = $\frac{18}{36}$ = $\frac{1}{2}$.
	\begin{flushright}	
		\textbf{Ответ:} $\frac{1}{2}$
	\end{flushright}
	\newpage
	\noindent
	\textbf{Задание 4.} 
	\\
	\textbf{Решение:} Рассмотрим сначала количество убывающих последовательностей длины 5 из чисел от 1 до 36. Выберем 5 чисел из 36. И просто упорядочим их по убыванию. Количество таких способов -- $C{_{36}^{5}}$.
	\\
	Теперь рассмотрим количество таких последовательностей, у которых в конце стоит единичка. Зафиксируем единичку в конце и выберем из оставшихся 35 чисел выберем 4 числа, упорядочим их по убыванию. Количество таких способов -- $C{_{35}^{4}}$.
	\\
	Искомая вероятность -- $\frac{C{_{35}^{4}}}{C{_{36}^{5}}}$ = $\frac{5}{36}$.
	\begin{flushright}
		\textbf{Ответ:} $\frac{5}{36}$
	\end{flushright}
	\textbf{Задание 5.} 
	\\
	\textbf{Решение:} Рассмотрим сначала неубывающие последовательности длины $5$, состоящие из целых чисел от $1$ до $36$. Нам нужно выбрать $5$ чисел из 36, учитывая повторы, количество таких последовательностей -- сочетания с повторениями из 36 по 5 т.е. $\overline{C}_{36}^{5} = C_{40}^{5}$ \\
	Теперь рассмотрим неубывающие последовательности длины $5$, состоящие из целых чисел от $1$ до $36$ и начинающиеся с $"1"$. Просто зафиксируем эту единичку и выберем 4 числа из 36, учитывая повторения. Их количество -- $\overline{C}_{36}^{4} = C_{39}^{4}$ \\
	В итоге искомая вероятность -- $\frac{C{_{39}^{4}}}{C{_{40}^{5}}} = \frac{1}{8}$
	
	\begin{flushright}
		\textbf{Ответ:} $\frac{1}{8}$
	\end{flushright}
	\textbf{Задание 6.} 
	\\
	\textbf{Решение:} Для начала найдем количество всех последовательностей длины 21, состоящих из 0 и 1. Нетрудно догадаться, что всего их -- $2^{21}$. \\
	Теперь найдем количество таких последовательностей, у которых на первых 10 позициях единичек стоит меньше чем на 11 последних. Это можно сделать двумя способами: \\
	1) Переберем количество единичек на последних 11 позициях, их может быть от 1 до 11. Пусть их $i$, тогда последовательностей длины 11, состоящих из 0 и 1, содержащих $i$ единичек -- $C_{11}^{i}$. Для такой каждой последовательности переберем количество единичек, стоящих на первых 10 позициях, их может быть от 0 до $i-1$. \\
	Итого, всего таких последовательностей -- $\sum\limits_{i=1}^{11}(C_{11}^{i} \cdot (\sum\limits_{j=0}^{i-1} C_{10}^{j}))$ = 1048576.
	Получается, искомая вероятность равна $\frac{1048576}{2^{21}} = \frac{1048576}{2097152} = \frac{1}{2}$. \\
	2) Этот способ почти не содержит сложных вычислений.\\
	Рассмотрим количество последовательностей, у которых в первых 10 позициях количество единичек меньше, чем в последних 10 позициях.
	Заметим, что последовательностей, где в первых 10 позициях количество единичек больше, чем в последних 10 позициях ровно столько же. Теперь рассмотрим, когда количество единичек в первых 10 позициях равно количеству в последних 10 позициях. У нас остается только одна позиция посередине, там может стоять либо 0, либо 1. Если стоит 0, то такая последовательность нам не подходит, если 1, то подходит. Значит среди последовательностей, где равное количество в первых и последних 10 позициях, нам подходит ровно половина.\\
	Получилось, что нам подходит ровно половина всевозможных последовательностей.
	
	\begin{flushright}
		\textbf{Ответ:} $\frac{1}{2}$
	\end{flushright}
	\textbf{Задание 7.} 
	\\
	\textbf{Решение:} Введем вероятностное пространство -- множество всех наборов из $n$ карт, где $n$ - целое число от 0 до 36. \\
	Нам нужно найти минимальное $n$, такое, что при вытаскивании $n$ карт из колоды вероятность появления туза больше 0.5.
	Найдем вероятность вероятность появления туза при вытаскивании $n$ карт из колоды. \\
	Всего вариантов различных наборов из $n$ карт, вытащенных из колоды -- $C_{36}^{n}$.\\
	Теперь рассмотрим количество наборов, не содержащих туза -- это $C_{32}^{n}$(колода превращается в 32 карты, так как мы выкидываем из нее четыре туза), значит количество наборов, содержащих туза -- $C_{36}^{n} - C_{32}^{n}$. \\
	Вероятность того, что вытащив $n$ карт, появится туз равна $\frac{C_{36}^{n} - C_{32}^{n}}{C_{36}^{n}} = 1 - \frac{C_{32}^{n}}{C_{36}^{n}}$. Это число должно быть больше 0.5 $\Rightarrow$ $\frac{C_{32}^{n}}{C_{36}^{n}} < 0.5$.\\
	Переберем $n$, и получим, что при $n=6$ $\frac{C_{32}^{n}}{C_{36}^{n}} < 0.5$, при $n < 6$ это сравнение нарушается! \\
	Значит 6 -- искомый ответ.
	
	\begin{flushright}
		\textbf{Ответ:} 6
	\end{flushright}
	\textbf{Задание 8.} 
	\\
	\textbf{Решение:} Пусть дни рождения распределены равномерно.\\
	Для начала посчитаем вероятность того, что у всех 30 человек дни рождения не совпадают. \\
	Для этого выберем случайно какого-либо человека и запомним его день рождения, далее выберем случайно второго, вероятность того, что у него день рождения не совпадёт с днем рождения первого человека, равна $1 - {\frac{1}{365}}$, затем выберем третьего человека, вероятность того, что его д.р. не совпадет с д.р. первых двух людей - $1 - {\frac{2}{365}}$. Рассуждая по аналогии, мы дойдём до 30-ого человека, для которого вероятность несовпадения его д.р. со всеми предыдущими будет равна $1 - {\frac{29}{365}}$.
	Так как нам надо, чтобы все эти условия выполнялись одновременно, то перемножим их вероятности и получим искомую вероятность: \\
	$(1 - {\frac{1}{365}}) \cdot (1 - {\frac{2}{365}}) \cdot ... \cdot (1 - {\frac{29}{365}}) = \frac{364 \cdot 363 \cdot ... \cdot (365 - 29)}{365^{29}} =  { 365! \over 365^{30} (365-30)! }$.\\
	Теперь, вероятность того, что у двух человек дни рождения совпадут равна \\ 
	$1 - { 365! \over 365^{30} (365-30)!}$. Если ее посчитать, то: \\ 
	$1 - { 365! \over 365^{30} (365-30)!} > 0.5$.
	\begin{flushright}
		\textbf{ч.т.д.}
	\end{flushright}
\end{document}
