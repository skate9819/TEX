\documentclass[12pt,a4paper]{scrartcl}
\usepackage[utf8]{inputenc}
\usepackage[english,russian]{babel}
\usepackage{indentfirst}
\usepackage{misccorr}
\usepackage{graphicx}
\usepackage{amsmath}

\begin{document}
	\begin{center}	
		Домашнне задание №18 \\
		Григорьев Дмитрий БПМИ--163
	\end{center}
	\textbf{Задание 1.}
	\newline
	\textbf{Решение:}
	\newline
	\indent
	Посмотрим на таблицу истинности для функции $x \oplus y \oplus z$. \\
	Заметим, что функция истинна, если ровно один из аргументов истинен или все три сразу. \\
	Тогда можно заменить функцию $x \oplus y \oplus z$ на $ \overline{(x \land (y \lor z) \lor y \land (x \lor z) \lor z \land (x \lor y))} \land (x \lor y \lor z)$. \\
	То есть мы получили функцию, использующую только одно отрицание, значит возможно построить необходимую схему.
	\begin{flushright}	
		\textbf{ч.т.д.}
	\end{flushright}
	\textbf{Задание 2.} 
	\newline
	\textbf{Решение:}
	\newline
	\indent
	Выпишем последние 9 наборов переменных. Заметим, что в 8 последних из них $x_1 = 1$, а в оставшемся девятом наборе $x_1 = 0, x_2 = 1, x_3 = 1, x_4 = 1$. \\
	Поэтому нам достаточно построить схему для функции $x_1 \lor (x_2 \land x_3 \land x_4)$. \\
	Схема:\\
	$x_1, x_2, x_3, x_4, x_2 \land x_3, (x_2 \land x_3) \land x_4, x_1 \lor ((x_2 \land x_3) \land x_4)$
	\begin{flushright}	
		\textbf{}
	\end{flushright}
	\textbf{Задание 3.} 
	\newline
	\textbf{Решение:} 
	\newline
	\indent
	Нам на вход подается $n$ битов. Нужно проверить, что $x_1 = 1, x_3 = 1$, а все остальные равны 0. \\
	Схема:\\
	$x_1, x_2, ... , x_n, \neg x_2, \neg x_4, \neg x_5, ... , \neg x_n, x_1 \land \neg x_2 , (x_1 \land \neg x_2) \land x_3, ((x_1 \land \neg x_2) \land x_3) \land \neg x_4, ... , x_1 \land \neg x_2 \land x_3 \land \neg x_4 \land ... \land \neg x_n$
	\begin{flushright}
		\textbf{}
	\end{flushright}
	\noindent
	\textbf{Задание 4.} 
	\newline
	\textbf{Решение:} 
	\newline
	\indent
	Нам нужно умножить двоичное число на 3. Пусть число длины $n$. Тогда нам необходимо умножить исходное число $x$ на 2 и прибавить к результату $x$. \\
	Чтобы умножить число $x$ на 2 нужно дописать к двоичной записи $x$ '0' справа. Т.е. $x_0 = 0$, так же для удобства запишем в $y$ исходное число $x$ с дописанным '0' слева, этот ноль ничего не изменит, но будет удобнее складывать 2 числа длины $n + 1$, получаем:
	$y_0 = x_1, y_1 = x_2, ... , y_{n - 1} = x_n, y_n = 0$. Теперь нужно построить схему сложения этих 2 чисел длины $n + 1$. \\
	Мы хотим построить схему с $2\cdot(n + 1)$ входами и $n + 2$ выходами.
	Идея конструкции схемы будет та же, что и в обычном школьном сложении в столбик. Мы будем складывать числа $x$ и $y$ поразрядно, попутно вычисляя биты переноса в следующий разряд. Для удобства будем обозначать через $b_i$ бит, который переносится в $i$-ый разряд
	из предыдущих. \\
	Введем дополнительные схемы: \\
	\textbullet Схема для $MAJ_3$: \\
	$x_1, x_2, x_3, x_1 \lor x_2,(x_1 \lor x_2) \land x_3,(x_1 \land x_2),((x_1 \lor x_2) \land x_3) \lor (x_1 \land x_2)$ \\
	\textbullet Схема для $xor$ от двух переменных: \\
	$x_1, x_2, \overline{x_1}, \overline{x_2}, \overline{x_1} \land x_2, x_1 \land \overline{x_2}, (\overline{x_1} \land x_2) \lor (x_1 \land \overline{x_2})$ \\ 
	Будем сохранять ответ в переменную $z$. $z_0$ можно посчитать сразу $z_0 = x_0 \oplus y_0$, далее видно, что $b_1 = MAJ_3(x_0, y_0, 0)$, $z_1 = x_1 \oplus y_1 \oplus b_1$, далее для любых $z_i, b_i$ вычисления будут такие же как и для $z_1$ и $b_1$, только в вычислении $b_i$ заменить в $MAJ_3$ '0' на $b_{i - 1}$.\\
	Схема:\\
	$x_0, x_1, ... , x_n, y_0, y_1, ... , y_n, b_1 = MAJ_3(x_0, y_0, 0), b_2 = MAJ_3(x_1, y_1, b_1), ... , b_{n} = MAJ_3(x_{n - 1}, y_{n - 1}, b_{n - 1}), \\ x_1 \oplus y_1, ... , x_n \oplus y_n, \underline {x_0 \oplus y_0, (x_1 \oplus y_1) \oplus b_1, ... , (x_n \oplus y_n) \oplus b_n, MAJ_3(x_n, y_n, b_n).}$ \\
	\\
	$n + 2$ выхода подчеркнуты в схеме.
	\begin{flushright}
	\textbf{}
	\end{flushright}
	\noindent
	\textbf{Задание 5.}
	\newline
	\textbf{Решение:}
	\newline
	\indent
	Нам нужно проверить, делится ли число на 3. Для этого достаточно рассмотреть знакочередующуюся сумму, а точнее ее делимость на 3(Так как мы можем воспользоваться следующим признаком делимости: Если основание системы счисления равно k - 1 по модулю некоторого числа k, то любое число делится на k тогда и только тогда, когда сумма цифр, занимающих нечётные места, либо равна сумме цифр, занимающих чётные места, либо отличается от неё на число, делящееся на k без остатка).\\
	Теперь, для того, чтобы посчитать остаток от деления на 3 знакочередующейся суммы, мы будем последовательно идти по всем битам, начиная с младшего разряда, и поддерживать остаток от деления на 3 текущей суммы. Для этого рассмотрим четные и нечетные позиции: \\
	\textbullet  Четные позиции \\
	\indent
		Если текущий бит равен '0', то остаток не изменится, если же '1', тогда 0 перейдет в 1, 1 перейдет в 2, 2 перейдет в 0. \\
		Теперь необходимо как то поддерживать остатки. Будем хранить их в виде двоичного числа, младший разряд которого на $i$-ом шаге равен $a_i$, а старший $b_i$, тогда 00, 01, 10 должно переходить в 01, 10, 00, если $x_i = 1$, и 00, 01, 10 остаться таким же, если $x_i = 0$.\\
		Заметим, что $b_i = a_{i - 1}$, а $a_i = \overline{a_{i - 1} \lor b_{i - 1}}$, только мы не учли, что при $x_i = 0$ мы не должны пересчитывать остаток. Получится:\\
		$b_i = a_{i - 1} \land x_i \lor b_{i - 1} \land \overline{x_i}$\\
		$a_i = \overline{a_{i - 1} \lor b_{i - 1}} \land x_i \lor a_{i - 1} \land \overline{x_i}$
	\newpage 
	\noindent
	\textbullet  Нечетные позиции \\
	\indent
	Если текущий бит равен '0', то остаток не изменится, если же '1', тогда 0 перейдет в 2, 1 перейдет в 0, 2 перейдет в 1. \\
	Остатки должны пересчитаться из 00, 01, 10 в 10, 00, 01, пересчитывать будем так: \\
	$b_i = \overline{a_{i - 1} \lor b_{i - 1}} \land x_i \lor b_{i - 1} \land \overline{x_i}$ \\
	$a_i = b_{i - 1} \land x_i \lor  a_{i - 1} \land \overline{x_i} $\\
	\\
	Теперь мы научились пересчитывать остатки, нужно указать только, что $a_0 = 0, b_0 = x_0$. Ответом будет $\overline{a_n \lor b_n}$.
\end{document}
