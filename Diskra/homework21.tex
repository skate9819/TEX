\documentclass[12pt,a4paper]{scrartcl}
\usepackage[utf8]{inputenc}
\usepackage[english,russian]{babel}
\usepackage{indentfirst}
\usepackage{misccorr}
\usepackage{graphicx}
\usepackage{amsmath}

\begin{document}
	\begin{center}	
		Домашнне задание №21 \\
		Григорьев Дмитрий БПМИ--163
	\end{center}
	\textbf{Задание 1.}
	\newline
	\textbf{Решение:}
	\newline
	\indent
	У нас $p \in \mathbb{N}$, и мы можем найти такое $y$, что $y \in \mathbb{N}, U(p, p) = y$, потому что мы можем выписать все алгоритмы, и найдется такой, что он вычисляет функцию $f(x) = y$. Так как мы выписали все алгоритмы, то найдется такой, что для любого $x, f(x) = y$. Тогда получается, что в независимости от $x, U(p, x) = y$. Получается, что $\forall y \in \mathbb{N}$ существует $p \in \mathbb{N}$, такое что $U(p, p) = y$. Следовательно заданное множество совпадает с $\mathbb{N}$.
	\begin{flushright}	
		\textbf{ч.т.д.}
	\end{flushright}
	\textbf{Задание 2.} 
	\newline
	\textbf{Решение:}
	\newline
	\indent
	a) Предположим, что данное множество разрешимо, тогда введем функцию $f(p) = U(p, p^2) + 1$. Она будет вычислима, так как вычислима $U$.  Теперь введем новую функцию $g(p)$, которая проверяет является ли $p$ точным квадратом: если да, то $g(p) = \sqrt{p}$, а иначе равна 0. Тогда $g(p)$ -- вычислима и $\exists n, U(n, p) = g(p) , \forall{p}$. Тогда $U(x, x^2) = g(x^2) = f(x) = U(x, x^2) + 1$.\\
	Противоречие! Следовательно данное множество неразрешимо.
	\begin{flushright}	
		\textbf{Ответ: да}
	\end{flushright}
	\indent
	б) Я утверждаю что нет, для этого покажем такую универсальную функцию. Пусть $U(p, x)$ равна 0, если $p$ -- точный квадрат, иначе равна $G(f(p), x)$, где $G$ -- некоторая универсальная вычислимая функция, $f(p)$ -- номер числа $p$ в списке всех не квадратов: 2, 3, 5, 6, 7, ... . Множество значений $f(p)$ совпадает с $\mathbb{N}$, следовательно $U$ -- действительно универсальная функция. Так как для любого квадрата $U(p, x) = 0$, то $U(p^2, p)$ -- определено. И исходное множество совпадает с $\mathbb{N}$. \\
	\begin{flushright}	
		\textbf{Ответ: нет}
	\end{flushright}
	\textbf{Задание 3.} 
	\newline
	\textbf{Решение:} 
	\newline
	\indent
	Рассмотрим бесконечное разрешимое множество натуральных чисел $A$. Члены этого множества возрастают, т.е. $f(n) = x_n$ -- вычислима и возрастает. \\
	Теперь рассмотрим перечислимое, неразрешимое подмножество в A -- B. Так как B -- перечеслимо, то $f(b)$ -- перечислимо , где $b \in B$. Если бы множество $\{ f(b) | b \in B \}$ было разрешимо, то мы бы могли проверить, что $f(n) \in f(B)$, а значит мы могли бы проверить, что $n \in X$ с учетом возрастания $f$, значит $X$ было бы пазрешимо. Противоречие.
	\begin{flushright}
		\textbf{ч.т.д.}
	\end{flushright}
	\noindent
	\textbf{Задание 4.}
	\newline
	\textbf{Решение:}
	\newline
	\indent
	Рассмотрим бесконечное разрешимое подмножество $\mathbb{N}$ -- $A$. Тогда мы можем в порядке возрастания идти по натуральным числам и выводить те, которые лежат в $A$ -- первое встретившееся $a_1$, второе $a_2$, n-ое $a_n$. Тогда мы получим $f(n) = a_n$, причем эта функция вычислима и возрастает. \\
	Теперь мы можем перечислить множество значений этой функции в порядке возрастания, следовательно $A$ -- разрешимо.
	\begin{flushright}
		\textbf{ч.т.д.}
	\end{flushright}
	
	\noindent
	\textbf{Задание 5.}
	\newline
	\textbf{Решение:}
	\newline
	\indent
	Будем $A$ -- некоторое бесконечное перечислимое множество. $B$ -- искомое бесконечное разрешимое подмножество. Тогда будем перечислять $A$ и добавлять в $B$ элемент из $A$ только в том случае, если очередной элемент $a_i$ больше чем максимальный из $B$. Таким образом мы получим возрастающую подпоследовательность $A$. Значит мы можем перечислить $B$ в порядке возрастания, следовательно $B$ -- разрешимо.
	\begin{flushright}
		\textbf{ч.т.д.}
	\end{flushright}
	\textbf{Задание 6.}
	\newline
	\textbf{Решение:}
	\newline
	\indent
	Пусть у нас есть область определения функции $U$ -- $F$, где $F$ -- множество пар $\{(p, x) | p, x \in \mathbb{N}\}$. $F$ перечеслимо, так как явзляется областью определения вычислимой функции. Тогда чтобы перечислить S, просто перечислим $F$, при этом выписывая не пару, а только первое число. Таким образом мы перечислим $S$.
	\begin{flushright}
		\textbf{ч.т.д.}
	\end{flushright}
\end{document}
