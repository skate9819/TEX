\documentclass[12pt,a4paper]{scrartcl}
\usepackage[utf8]{inputenc}
\usepackage[english,russian]{babel}
\usepackage{indentfirst}
\usepackage{misccorr}
\usepackage{graphicx}
\usepackage{amsmath}

\begin{document}
	\begin{center}	
		Домашнне задание №2 \\
		Григорьев Дмитрий БПМИ--163
	\end{center}
	\textbf{Задание 1.}
	\newline
	\textbf{Решение:}
	\newline
	\indent
	
	Найдем для начала все левые смежные классы, для этого каждую четную перестановку умножим слева на все элементы $\langle\sigma\rangle$(а их всего 2):  \\
	$\bullet$
	$\begin{pmatrix}
		1~2~3~4\\
		1~2~3~4
	\end{pmatrix}$
	$\begin{pmatrix}
	1~2~3~4\\
	2~1~4~3
	\end{pmatrix} = $ $\begin{pmatrix}
	1~2~3~4\\
	2~1~4~3
	\end{pmatrix}$, $\begin{pmatrix}
	1~2~3~4\\
	1~2~3~4
	\end{pmatrix}$
	$\begin{pmatrix}
	1~2~3~4\\
	1~2~3~4
	\end{pmatrix} = $ $\begin{pmatrix}
	1~2~3~4\\
	1~2~3~4
	\end{pmatrix}$\\
	$\bullet$
	$\begin{pmatrix}
	1~2~3~4\\
	1~3~4~2
	\end{pmatrix}$
	$\begin{pmatrix}
	1~2~3~4\\
	2~1~4~3
	\end{pmatrix} = $ $\begin{pmatrix}
	1~2~3~4\\
	3~1~2~4
	\end{pmatrix}$, $\begin{pmatrix}
	1~2~3~4\\
	1~3~4~2
	\end{pmatrix}$
	$\begin{pmatrix}
	1~2~3~4\\
	1~2~3~4
	\end{pmatrix} = $ $\begin{pmatrix}
	1~2~3~4\\
	1~3~4~2
	\end{pmatrix}$\\
	$\bullet$
	$\begin{pmatrix}
	1~2~3~4\\
	1~4~2~3
	\end{pmatrix}$
	$\begin{pmatrix}
	1~2~3~4\\
	2~1~4~3
	\end{pmatrix} = $ $\begin{pmatrix}
	1~2~3~4\\
	4~1~3~2
	\end{pmatrix}$, 
	$\begin{pmatrix}
	1~2~3~4\\
	1~4~2~3
	\end{pmatrix}$
	$\begin{pmatrix}
	1~2~3~4\\
	1~2~3~4
	\end{pmatrix} = $
	$\begin{pmatrix}
	1~2~3~4\\
	1~4~2~3	\end{pmatrix}$\\
	$\bullet$
	$\begin{pmatrix}
	1~2~3~4\\
	2~1~4~3
	\end{pmatrix}$
	$\begin{pmatrix}
	1~2~3~4\\
	2~1~4~3
	\end{pmatrix} = $ $\begin{pmatrix}
	1~2~3~4\\
	1~2~3~4
	\end{pmatrix}$, 
	$\begin{pmatrix}
	1~2~3~4\\
	2~1~4~3
	\end{pmatrix}$
	$\begin{pmatrix}
	1~2~3~4\\
	1~2~3~4
	\end{pmatrix} = $
	$\begin{pmatrix}
	1~2~3~4\\
	2~1~4~3	\end{pmatrix}$\\
	$\bullet$
	$\begin{pmatrix}
	1~2~3~4\\
	2~3~1~4
	\end{pmatrix}$
	$\begin{pmatrix}
	1~2~3~4\\
	2~1~4~3
	\end{pmatrix} = $ $\begin{pmatrix}
	1~2~3~4\\
	3~2~4~1
	\end{pmatrix}$, 
	$\begin{pmatrix}
	1~2~3~4\\
	2~3~1~4
	\end{pmatrix}$
	$\begin{pmatrix}
	1~2~3~4\\
	1~2~3~4
	\end{pmatrix} = $
	$\begin{pmatrix}
	1~2~3~4\\
	2~3~1~4	\end{pmatrix}$\\
	$\bullet$
	$\begin{pmatrix}
	1~2~3~4\\
	2~4~3~1
	\end{pmatrix}$
	$\begin{pmatrix}
	1~2~3~4\\
	2~1~4~3
	\end{pmatrix} = $ $\begin{pmatrix}
	1~2~3~4\\
	4~2~3~1
	\end{pmatrix}$, 
	$\begin{pmatrix}
	1~2~3~4\\
	2~4~3~1
	\end{pmatrix}$
	$\begin{pmatrix}
	1~2~3~4\\
	1~2~3~4
	\end{pmatrix} = $
	$\begin{pmatrix}
	1~2~3~4\\
	2~4~3~1	\end{pmatrix}$\\
	$\bullet$
	$\begin{pmatrix}
	1~2~3~4\\
	3~1~2~4
	\end{pmatrix}$
	$\begin{pmatrix}
	1~2~3~4\\
	2~1~4~3
	\end{pmatrix} = $ $\begin{pmatrix}
	1~2~3~4\\
	1~3~4~2
	\end{pmatrix}$, 
	$\begin{pmatrix}
	1~2~3~4\\
	3~1~2~4
	\end{pmatrix}$
	$\begin{pmatrix}
	1~2~3~4\\
	1~2~3~4
	\end{pmatrix} = $
	$\begin{pmatrix}
	1~2~3~4\\
	3~1~2~4	\end{pmatrix}$\\
	$\bullet$
	$\begin{pmatrix}
	1~2~3~4\\
	3~2~4~1
	\end{pmatrix}$
	$\begin{pmatrix}
	1~2~3~4\\
	2~1~4~3
	\end{pmatrix} = $ $\begin{pmatrix}
	1~2~3~4\\
	2~3~1~4
	\end{pmatrix}$, 
	$\begin{pmatrix}
	1~2~3~4\\
	3~2~4~1
	\end{pmatrix}$
	$\begin{pmatrix}
	1~2~3~4\\
	1~2~3~4
	\end{pmatrix} = $
	$\begin{pmatrix}
	1~2~3~4\\
	3~2~4~1	\end{pmatrix}$\\
	$\bullet$
	$\begin{pmatrix}
	1~2~3~4\\
	3~4~1~2
	\end{pmatrix}$
	$\begin{pmatrix}
	1~2~3~4\\
	2~1~4~3
	\end{pmatrix} = $ $\begin{pmatrix}
	1~2~3~4\\
	4~3~2~1
	\end{pmatrix}$, 
	$\begin{pmatrix}
	1~2~3~4\\
	3~4~1~2
	\end{pmatrix}$
	$\begin{pmatrix}
	1~2~3~4\\
	1~2~3~4
	\end{pmatrix} = $
	$\begin{pmatrix}
	1~2~3~4\\
	3~4~1~2	\end{pmatrix}$\\
	$\bullet$
	$\begin{pmatrix}
	1~2~3~4\\
	4~1~3~2
	\end{pmatrix}$
	$\begin{pmatrix}
	1~2~3~4\\
	2~1~4~3
	\end{pmatrix} = $ $\begin{pmatrix}
	1~2~3~4\\
	1~4~2~3
	\end{pmatrix}$, 
	$\begin{pmatrix}
	1~2~3~4\\
	4~1~3~2
	\end{pmatrix}$
	$\begin{pmatrix}
	1~2~3~4\\
	1~2~3~4
	\end{pmatrix} = $
	$\begin{pmatrix}
	1~2~3~4\\
	4~1~3~2	\end{pmatrix}$\\
	$\bullet$
	$\begin{pmatrix}
	1~2~3~4\\
	4~2~1~3
	\end{pmatrix}$
	$\begin{pmatrix}
	1~2~3~4\\
	2~1~4~3
	\end{pmatrix} = $ $\begin{pmatrix}
	1~2~3~4\\
	2~4~3~1
	\end{pmatrix}$, 
	$\begin{pmatrix}
	1~2~3~4\\
	4~2~1~3
	\end{pmatrix}$
	$\begin{pmatrix}
	1~2~3~4\\
	1~2~3~4
	\end{pmatrix} = $
	$\begin{pmatrix}
	1~2~3~4\\
	4~2~1~3	\end{pmatrix}$\\
	$\bullet$
	$\begin{pmatrix}
	1~2~3~4\\
	4~3~2~1
	\end{pmatrix}$
	$\begin{pmatrix}
	1~2~3~4\\
	2~1~4~3
	\end{pmatrix} = $ $\begin{pmatrix}
	1~2~3~4\\
	3~4~1~2
	\end{pmatrix}$, 
	$\begin{pmatrix}
	1~2~3~4\\
	4~3~2~1
	\end{pmatrix}$
	$\begin{pmatrix}
	1~2~3~4\\
	1~2~3~4
	\end{pmatrix} = $
	$\begin{pmatrix}
	1~2~3~4\\
	4~3~2~1	\end{pmatrix}$\\
	Как видим некоторые из них совпадают, поэтому уберем лишние:\\
	1. 	$\begin{pmatrix}
	1~2~3~4\\
	2~1~4~3	\end{pmatrix}$, 
		$\begin{pmatrix}
	1~2~3~4\\
	1~2~3~4	\end{pmatrix}$ \\
	2. 	$\begin{pmatrix}
	1~2~3~4\\
	3~1~2~4	\end{pmatrix}$, 
	$\begin{pmatrix}
	1~2~3~4\\
	1~3~4~2	\end{pmatrix}$ \\
	3. 	$\begin{pmatrix}
	1~2~3~4\\
	4~1~3~2	\end{pmatrix}$, 
	$\begin{pmatrix}
	1~2~3~4\\
	1~4~2~3	\end{pmatrix}$ \\
	4. 	$\begin{pmatrix}
	1~2~3~4\\
	3~2~4~1	\end{pmatrix}$, 
	$\begin{pmatrix}
	1~2~3~4\\
	2~3~1~4	\end{pmatrix}$ \\
	5. 	$\begin{pmatrix}
	1~2~3~4\\
	4~2~3~1	\end{pmatrix}$, 
	$\begin{pmatrix}
	1~2~3~4\\
	2~4~3~1	\end{pmatrix}$ \\
	6. 	$\begin{pmatrix}
	1~2~3~4\\
	4~3~2~1	\end{pmatrix}$, 
	$\begin{pmatrix}
	1~2~3~4\\
	3~4~1~2	\end{pmatrix}$ \\
	Теперь таким же образом найдем правые смежные классы, в итоге получится: \\
	1. 	$\begin{pmatrix}
	1~2~3~4\\
	2~1~4~3	\end{pmatrix}$, 
	$\begin{pmatrix}
	1~2~3~4\\
	1~2~3~4	\end{pmatrix}$ \\
	2. 	$\begin{pmatrix}
	1~2~3~4\\
	2~4~3~1	\end{pmatrix}$, 
	$\begin{pmatrix}
	1~2~3~4\\
	1~3~4~2	\end{pmatrix}$ \\
	3. 	$\begin{pmatrix}
	1~2~3~4\\
	2~3~1~4	\end{pmatrix}$, 
	$\begin{pmatrix}
	1~2~3~4\\
	1~4~2~3	\end{pmatrix}$ \\
	4. 	$\begin{pmatrix}
	1~2~3~4\\
	4~2~3~1	\end{pmatrix}$, 
	$\begin{pmatrix}
	1~2~3~4\\
	3~1~2~4	\end{pmatrix}$ \\
	5. 	$\begin{pmatrix}
	1~2~3~4\\
	4~1~3~2	\end{pmatrix}$, 
	$\begin{pmatrix}
	1~2~3~4\\
	3~2~4~1	\end{pmatrix}$ \\
	6. 	$\begin{pmatrix}
	1~2~3~4\\
	4~3~2~1	\end{pmatrix}$, 
	$\begin{pmatrix}
	1~2~3~4\\
	3~4~1~2	\end{pmatrix}$ \\	
	Подгруппа $H$ не является нормальной, так как для $g = \begin{pmatrix}
	1~2~3~4\\
	1~3~4~2
	\end{pmatrix}
	$ не выполняется условие $gH = Hg$
	\begin{flushright}	
		\textbf{}
	\end{flushright}
	\textbf{Задание 2.} 
	\newline
	\textbf{Решение:} 
	\newline
	\indent
	Множество $\{ \begin{pmatrix}
	a~b\\
	c~d
	\end{pmatrix} \in SL_2(\mathbb{Z}) |  a\equiv b\equiv1~(mod 3); b \equiv c\equiv 0~(mod 3)\}$ является нормальной подгруппой в $SL_2(\mathbb{Z})$ $<=>$ $gHg^{-1} \subseteq H$, где $g \in SL_2(\mathbb{Z})$, $H$ -- данное множество.\\
	
	Пусть $g = \begin{pmatrix}
	a~b\\
	c~d
	\end{pmatrix}$, тогда $g^{-1} = \frac{1}{det(g)} \begin{pmatrix}
	~~d~-b\\
	-c~~~a
	\end{pmatrix} = \begin{pmatrix}
	~~d~-b\\
	-c~~~a
	\end{pmatrix}$ \\
	
	Пусть $h \in H$,  $h = \begin{pmatrix}
	s~q\\
	r~t
	\end{pmatrix}$ \\
	
	Тогда $ghg^{-1} = \begin{pmatrix}
	a~b\\
	c~d
	\end{pmatrix} \begin{pmatrix}
	s~q\\
	r~t
	\end{pmatrix}\begin{pmatrix}
	~~d~-b\\
	-c~~~a
	\end{pmatrix} = \\
	
	=\begin{pmatrix}
	-acq+bdr+ads-bct~~~a^2q-b^2r-abs+abt\\
	-c^2q+d^2r+cds-cdt~~~acq-bdr-bcs+adt
	\end{pmatrix} = \begin{pmatrix}
	1~0\\
	0~1
	\end{pmatrix} ~ (mod~3)$ \\
	
	Получилось, что $ghg^{-1} \subseteq H =>$ \\
	
	$=>$ множество  $\{ \begin{pmatrix}
	a~b\\
	c~d
	\end{pmatrix} \in SL_2(\mathbb{Z}) |  a\equiv b\equiv1~(mod 3); b \equiv c\equiv 0~(mod 3)\}$ является \\
	
	нормальной подгруппой в $SL_2(\mathbb{Z})$.

	\begin{flushright}
		\textbf{ч.т.д.}
	\end{flushright}

\end{document}
