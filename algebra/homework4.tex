\documentclass[12pt,a4paper]{scrartcl}
\usepackage[utf8]{inputenc}
\usepackage[english,russian]{babel}
\usepackage{indentfirst}
\usepackage{misccorr}
\usepackage{graphicx}
\usepackage{amsmath}

\begin{document}
	\begin{center}	
		Домашнне задание №4 \\
		Григорьев Дмитрий БПМИ--163
	\end{center}
	\textbf{Задание 1.}
	\\
	\textbf{Решение:}
	\\
	\indent
	$\bullet$ Пусть $x \in X$, тогда орбита $x~~Gx = \{g \cdot x | g \in G\}$ -- это элементы, которые мы можем получить действием $G$ на $x$.\\
	Пусть $g = \begin{pmatrix}
	a~0~0\\
	0~b~0\\
	0~0~c\\
	\end{pmatrix}$, а $x = \begin{pmatrix}
	q\\
	w\\
	e\\
	\end{pmatrix}$, где $a, b, c \in \mathbb{R} \setminus \{0\}~~~ q, w, e \in \mathbb{R}$\\
	Тогда $g \cdot x = \begin{pmatrix}
	a~0~0\\
	0~b~0\\
	0~0~c\\
	\end{pmatrix} \cdot \begin{pmatrix}
	q\\
	w\\
	e\\
	\end{pmatrix} = \begin{pmatrix}
	a\cdot q\\
	b \cdot w\\
	c \cdot e\\
	\end{pmatrix}$.\\
	Но $q, w$ или $e$ могут равняться нулю. Тогда все орбиты  для действия $G$ на множестве $X$ будут иметь вид:
	$$ \{ \begin{pmatrix}
	a\cdot q\\
	b \cdot w\\
	c \cdot e\\
	\end{pmatrix}\}, \{ \begin{pmatrix}
	0\\
	b \cdot w\\
	c \cdot e\\
	\end{pmatrix}\}, \{ \begin{pmatrix}
	a\cdot q\\
	0\\
	c \cdot e\\
	\end{pmatrix}\}, \{ \begin{pmatrix}
	a\cdot q\\
	b \cdot w\\
	0\\
	\end{pmatrix}\}, \{ \begin{pmatrix}
	a\cdot q\\
	0\\
	0\\
	\end{pmatrix}\}, \{ \begin{pmatrix}
	0\\
	b \cdot w\\
	0\\
	\end{pmatrix}\}, \{ \begin{pmatrix}
	0\\
	0\\
	c \cdot e\\
	\end{pmatrix}\}, \{ \begin{pmatrix}
	0\\
	0\\
	0\\
	\end{pmatrix}\}$$
	\\
	$\bullet$ Пусть $x \in X$, стабилизатор $x~St(x) := \{g \in G | g \cdot x = g\}$ -- это элементы $G$, которые не изменяют $x$ при действии.\\
	Пусть $g = \begin{pmatrix}
	a~0~0\\
	0~b~0\\
	0~0~c\\
	\end{pmatrix}$ -- стабилизатор $x$, а $x = \begin{pmatrix}
	q\\
	w\\
	e\\
	\end{pmatrix}$, где $a, b, c \in \mathbb{R} \setminus \{0\}~~~ q, w, e \in \mathbb{R}$\\
	Тогда $g \cdot x = \begin{pmatrix}
	a~0~0\\
	0~b~0\\
	0~0~c\\
	\end{pmatrix} \cdot \begin{pmatrix}
	q\\
	w\\
	e\\
	\end{pmatrix} = \begin{pmatrix}
	a\cdot q\\
	b \cdot w\\
	c \cdot e\\
	\end{pmatrix} = \begin{pmatrix}
	q\\
	w\\
	e\\
	\end{pmatrix}$\\
	$\begin{cases}
	aq = q\\
	bw = w\\
	ce = e
	\end{cases} \Rightarrow \begin{cases}
	a = \begin{cases}
	1, q \neq 0 \\
	a \in \mathbb{R} \setminus \{0\}, q = 0\\
	\end{cases} \\
	b = \begin{cases}
	1, w \neq 0 \\
	b \in \mathbb{R} \setminus \{0\}, w = 0\\
	\end{cases} \\
	c = \begin{cases}
	1, e \neq 0 \\
	c \in \mathbb{R} \setminus \{0\}, e = 0\\
	\end{cases} \\
		\end{cases}$\\
	Тогда все стабилизаторы для действия $G$ на множестве $X$ будут иметь вид:
	$$ \{ \begin{pmatrix}
	x~~0~~0\\
	0~~y~~0\\
	0~~0~~z\\
	\end{pmatrix}\}, \{ \begin{pmatrix}
	1~~0~~0\\
	0~~y~~0\\
	0~~0~~z\\
	\end{pmatrix}\}, \{ \begin{pmatrix}
	x~~0~~0\\
	0~~1~~0\\
	0~~0~~z\\
	\end{pmatrix}\}, \{ \begin{pmatrix}
	x~~0~~0\\
	0~~y~~0\\
	0~~0~~1\\
	\end{pmatrix}\}, \{ \begin{pmatrix}
	1~~0~~0\\
	0~~1~~0\\
	0~~0~~z\\
	\end{pmatrix}\}, $$ $$\{ \begin{pmatrix}
	1~~0~~0\\
	0~~y~~0\\
	0~~0~~1\\
	\end{pmatrix}\}, \{ \begin{pmatrix}
	x~~0~~0\\
	0~~1~~0\\
	0~~0~~1\\
	\end{pmatrix}\}, \{ \begin{pmatrix}
	1~~0~~0\\
	0~~1~~0\\
	0~~0~~1\\
	\end{pmatrix}\}~~~x, y, z \in \mathbb{R} \setminus \{0\}$$
	\begin{flushright}	
		\textbf{}
	\end{flushright}
	\textbf{Задание 2.}\\
	\textbf{Решение:} 
	\\
	\indent
	Пусть $x = \begin{pmatrix}
	a~~b\\
	0~~\frac{1}{a}
	\end{pmatrix}, g = \begin{pmatrix}
	q~~w\\
	0~~\frac{1}{q}
	\end{pmatrix}~~~~x, g \in G$\\
	Тогда действие сопряжением равно:\\
	$xgx^{-1} = \begin{pmatrix}
	a~~b\\
	0~~\frac{1}{a}
	\end{pmatrix} \begin{pmatrix}
	q~~w\\
	0~~\frac{1}{q}
	\end{pmatrix} \begin{pmatrix}
		\frac{1}{a}~~-b\\
		0~~~~~a
	\end{pmatrix} = \begin{pmatrix}
	q&\frac{ab-abq^2+a^2qw}{q}\\
	0&q
	\end{pmatrix}$\\
	Теперь рассмотрим $\frac{ab-abq^2+a^2qw}{q} = a^2w+ab(\frac{1 - q^2}{q})$:\\
	1)$w \neq 0, \frac{1 - q^2}{1} \neq 0$, тогда $a^2w+ab(\frac{1 - q^2}{q}) \in \mathbb{R}$, и мы можем получить любое значение из $\mathbb{R}$\\
	2)$w = 0, \frac{1 - q^2}{q} \neq 0$, тогда 
	$a^2w+ab(\frac{1 - q^2}{q}) \in \mathbb{R}$, и мы можем получить любое значение из $\mathbb{R}$\\
	3)$w \neq 0, \frac{1 - q^2}{q} = 0$, тогда\\
	если $w < 0$, то  $a^2w+ab(\frac{1 - q^2}{q}) \in \mathbb{R}_-$, и мы можем получить любое значение из $\mathbb{R}_-$\\
	если $w > 0$, то  $a^2w+ab(\frac{1 - q^2}{q}) \in \mathbb{R}_+$, и мы можем получить любое значение из $\mathbb{R}_+$\\
	4)$w = 0, \frac{1 - q^2}{q} = 0$, тогда 
	$a^2w+ab(\frac{1 - q^2}{q}) = 0$\\
	Получается, что классы сопряжености будут иметь вид:\\
	$\left\{
	\begin{pmatrix}
	x&y\\
	0&\frac{1}{x}
	\end{pmatrix} \bigg| 
	 	y \in \mathbb{R}
	\right\}x \in \mathbb{R} \backslash \{-1, 0, 1\}, \\
	\left\{
	\begin{pmatrix}
	x&y\\
	0&\frac{1}{x}
	\end{pmatrix} \bigg| 
	y \in \mathbb{R}_-
	\right\}x \in \{-1, 1\},
	\\ \left\{
	\begin{pmatrix}
	x&y\\
	0&\frac{1}{x}
	\end{pmatrix} \bigg| 
	y \in \mathbb{R}_+
	\right\}x \in \{-1, 1\}, \\
	\left\{
	\begin{pmatrix}
	-1&0\\
	0&-1
	\end{pmatrix} \right\}, \left\{
	\begin{pmatrix}
	1&0\\
	0&1
	\end{pmatrix} \right\}$\\
	\textbf{Задание 3.} 
	\\
	\textbf{Решение:} 
	\\
	\indent
	$\sigma = (1, 2, 3, 4) = \begin{pmatrix}
	1, 2, 3, 4\\
	2, 3, 4, 1
	\end{pmatrix}$\\
	Стабилизатор $St(\sigma) = \{s \in S_4 | s\cdot \sigma \cdot s^{-1} = \sigma\}$ -- это элементы $S_4$, которые не изменяют $\sigma$ при действиями сопряжениями.\\
	Теперь рассмотрим $s \in St(\sigma)$. Посмотрим как действует сопряжением $s$:\\
	так как $s\cdot \sigma \cdot s^{-1} = \sigma \Rightarrow s\cdot \sigma = \sigma \cdot s$\\
	 $$s(\sigma(1)) = \sigma(s(1))$$
	 $$s(\sigma(2)) = \sigma(s(2))$$
	 $$s(\sigma(3)) = \sigma(s(3))$$
	 $$s(\sigma(4)) = \sigma(s(4))$$
	 $$\Downarrow$$
	 $$s(2) = \sigma(s(1))$$ 
	 $$s(3) = \sigma(s(2))$$
	 $$s(4) = \sigma(s(3))$$
	 $$s(1) = \sigma(s(4))$$
	 Видно, что $s$ зависит от какого то одного элемента , т.е. зная один элемент $s$ можно получить все остальные элементы. Поэтому можно перебрать какой-либо элемент $s$. Переберем $s(4)$:\\
	 1) $s(4) = 1 \Rightarrow
	 s(1) = \sigma(s(4)) = 2 \Rightarrow 
	 s(2) = \sigma(s(1)) = 3 \Rightarrow 
	 s(3) = \sigma(s(2)) = 4$\\
	 2) $s(4) = 2\Rightarrow 
	 s(1) = \sigma(s(4)) = 3 \Rightarrow 
	 s(2) = \sigma(s(1)) = 4 \Rightarrow 
	 s(3) = \sigma(s(2)) = 1$\\
	 3) $s(4) = 3\Rightarrow 
	 s(1) = \sigma(s(4)) = 4 \Rightarrow  
	 s(2) = \sigma(s(1)) = 1 \Rightarrow 
	 s(3) = \sigma(s(2)) = 2$\\
	 4) $s(4) = 4\Rightarrow 
	 s(1) = \sigma(s(4)) = 1 \Rightarrow  
	 s(2) = \sigma(s(1)) = 2 \Rightarrow 
	 s(3) = \sigma(s(2)) = 3$\\
	 Получается, что $St(\sigma) = \{
	 \begin{pmatrix}
	 1, 2, 3, 4\\
	 2, 3, 4, 1
	 \end{pmatrix}, \begin{pmatrix}
	 1, 2, 3, 4\\
	 3, 4, 1, 2
	 \end{pmatrix}, \begin{pmatrix}
	 1, 2, 3, 4\\
	 4, 1, 2, 3
	 \end{pmatrix}, id\}$\\
	\textbf{Задание 4.} \\
	\textbf{Решение:} 
	\\
	\indent
	Пусть $a \in \mathbb{Z}_k \times \mathbb{Z}_l$, тогда рассмотрим действие $a$ на элемент $g = (q, w)$.\\
	Если $a = (0, 1)$, тогда при действии $a$ на $g$ $w$ меняется по циклу длины $l$ а $q$ не меняется. Тогда четность такой перестановки равна $(-1)^{k(l - 1)}$(так как $k$ циклов длинны $l$) \\
	Если же $a = (1, 0)$, тогда при действии $a$ на $g$ $q$ меняется по циклу длины $k$ а $w$ не меняется. Тогда четность такой перестановки равна $(-1)^{l(k - 1)}$(так как $l$ циклов длинны $k$)\\
	Если же $a = (x, y)$ тогда этому элементу соответствует перестановка, которая выражается через $(1, 0)$ в степени $x$ и через $(0, 1)$ в степени $y$.\\
	Поэтому чтобы перестановка была четной должно выполнятся:\\
	$\begin{cases}
	k(l - 1)~mod ~2 = 0\\
	l(k - 1) ~mod ~2 = 0
	\end{cases}$ \\
	Тогда получается, что $k$ и $l$ должны быть одинаковой четности.
\end{document}
