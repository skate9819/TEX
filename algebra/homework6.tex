\documentclass[12pt,a4paper]{scrartcl}
\usepackage[utf8]{inputenc}
\usepackage[english,russian]{babel}
\usepackage{indentfirst}
\usepackage{misccorr}
\usepackage{graphicx}
\usepackage{amsmath}

\begin{document}
	\begin{center}	
		Домашнне задание №6 \\
		Григорьев Дмитрий БПМИ--163
	\end{center}
	\textbf{Задание 1.}
	\\
	\textbf{Решение:}
	\\
	$f(x) = 2x^4 - 4x^3 - 3x^2 + 7x - 2\\
	g(x) = 6x^3 + 4x^2 - 5x + 1$\\
	Воспользуемся алгоритмом Евклида.\\
	Сначала разложим многочлены на множители:\\
	$f(x) = (x - 1)(2x^3 - 2x^2 - 5x + 2) = (x - 1)(x - 2)(2x^2 + 2x - 1)$\\
	$g(x) = (2x^2 + 2x - 1)(3x - 1)$\\
	Так как $(2x^2 + 2x - 1)$ -- делитель $f(x)$ и $g(x)$, тогда он будет содержаться в НОД($f(x), g(x)$).\\
	Теперь найдем НОД $(x - 1)(x - 2)$ и $(3x - 1)$:\\
	Если разделим столбиком $(x - 1)(x - 2)$ на $(3x - 1)$, то получим:\\
	$(x - 1)(x - 2) = (3x - 1)(\frac{1}{3}x - \frac{8}{9}) + \frac{10}{9}$\\
	Если разделим столбиком $(3x - 1)$ на $\frac{10}{9}$, то получим:\\
	$(3x - 1) = \frac{10}{9}(\frac{27}{10}x - \frac{9}{10})$\\
	\\
	Получили, что НОД($f(x), g(x)$) = $\frac{10}{9}(2x^2 + 2x - 1)$\\
	Рассмотрим первое деление столбиком. Если домножим его на $(2x^2 + 2x - 1)$, то получим:\\
	$f(x) = g(x)(\frac{1}{3}x - \frac{8}{9})$ + НОД$(f(x), g(x)) \Rightarrow$ НОД$(f(x), g(x) = f(x) + g(x)(-\frac{1}{3}x + \frac{8}{9})$\\
	\\
	\textbf{Задание 2.}\\
	\textbf{Решение:} 
	\\
	$x^6 + x^3 - 12 = (x^3 + 4)(x^3 - 3) = (x + \sqrt[3]{4})(x^2 - \sqrt[3]{4}x + \sqrt[3]{16})(x - \sqrt[3]{3})(x^2 + \sqrt[3]{3}x + \sqrt[3]{9})$\\
	Это разложение в $\mathbb{R}[x]$.\\
	Тепрь разложим в $\mathbb{C}[x]$:\\
	\\
	1) $x^2 - \sqrt[3]{4}x + \sqrt[3]{16} = 0$\\
	$D = (\sqrt[3]{4})^2 - 4\sqrt[3]{16} = -3\sqrt[3]{16}$\\
	$x = \frac{\sqrt[3]{4} \pm \sqrt{-3\sqrt[3]{16}}}{2} = \frac{\sqrt[3]{4} \pm \sqrt{3}i\sqrt[3]{4}}{2} = \frac{\sqrt[3]{4}(1 \pm \sqrt{3}i)}{2}$\\
	\\
	2) $x^2 + \sqrt[3]{3}x + \sqrt[3]{9} = 0$\\
	$D = (\sqrt[3]{3})^2 - 4\sqrt[3]{9} = -3\sqrt[3]{9}$\\
	$x = \frac{-\sqrt[3]{3} \pm \sqrt{-3\sqrt[3]{9}}}{2} = \frac{-\sqrt[3]{3} \pm \sqrt{3}i\sqrt[3]{3}}{2} = \frac{-\sqrt[3]{3}(1 \pm \sqrt{3}i)}{2}$\\
	\\
	Разложение в $\mathbb{C}[x]:$\\
	$x^6 + x^3 - 12 = (x^3 + 4)(x^3 - 3) = (x + \sqrt[3]{4})(x^2 - \sqrt[3]{4}x + \sqrt[3]{16})(x - \sqrt[3]{3})(x^2 + \sqrt[3]{3}x + \sqrt[3]{9}) = (x + \sqrt[3]{4})(x - \frac{\sqrt[3]{4}(1 + \sqrt{3}i)}{2})(x - \frac{\sqrt[3]{4}(1 - \sqrt{3}i)}{2})(x - \sqrt[3]{3})(x - \frac{-\sqrt[3]{3}(1 + \sqrt{-3}i)}{2})(x - \frac{-\sqrt[3]{3}(1 - \sqrt{3}i)}{2})$\\
	\\
	\textbf{Задание 3.} 
	\\
	\textbf{Решение:} \\
	У нас есть $(5 + \sqrt{-5})$.\\
	Возьмем произведение $(1 + \sqrt{-5})(1 - \sqrt{-5})$, оно равно 6.\\
	Нужно свести как то к $(5 + \sqrt{-5})$. Умножим на $\sqrt{-5}$:
	$(1 + \sqrt{-5})(1 - \sqrt{-5}) \cdot \sqrt{-5} = (1 + \sqrt{-5})(5 + \sqrt{-5})$\\
	\\
	Получили:
	$$(1 + \sqrt{-5})(5 + \sqrt{-5}) = 6 \cdot \sqrt{-5}$$
	Тогда рассмотрим норму $f = a + bi\sqrt{5}$:\\
	$N(f) = a^2 + 5b^2$\\
	Тогда $N(5 + \sqrt{-5}) = 25 + 5 = 30$\\
	Так же распишем $6 \cdot \sqrt{-5}$ как $3 \cdot 2\sqrt{-5}$\\
	$N(3) = 9$\\
	$N(2\sqrt{-5}) = 20$\\
	\\
	Тогда, чтобы $(5 + \sqrt{-5})$ было простым элементом кольца $\mathbb{Z}[\sqrt{-5}]$, то ,так как $6\sqrt
	-5$ делится на $(5 + \sqrt{-5})$(из полученного ранее равенства), либо $2\sqrt{-5}$, либо $3$ делится на $(5 + \sqrt{-5})$.\\
	Но $N(3) < N(5 + \sqrt{-5})$ и 	$N(2\sqrt{-5}) < N(5 + \sqrt{-5})$.\\
	\\
	Получается, что $5 + \sqrt{-5}$ -- не простой элемент кольца $\mathbb{Z}[\sqrt{-5}]$.\\
	\\
	\textbf{Задание 4.} \\
	\textbf{Решение:} 
	\\
	 Пусть у нас найдется такой элемент $x$, что $N(x)$ -- максимально возможный, тогда так как мы знаем, что $N(ab) \geqslant N(a)$ и равно тогда и только тогда, когда $b$ обратим.\\
	 Тогда $N(x^2)$ не может превосходить $N(x)$(так как это максимально возможное значение), значит $x$ -- обратим.\\
	 Рассмотрим $N(x\cdot x^{-1})$. Эта норма равна $N(x)$ и так же равна $N(1)$.\\
	 \\
	 Любой ненулевой элемент $a$ мы можем представить в виде $(1\cdot a)$. Тогда рассмторим норму этого элемента:\\
	 $N(a) = N(1\cdot a) \geqslant N(1)$. Она не может превосходить $N(1)$, так как это максимальное знаечение, тогда получили, что норма любого ненулевого элемента равна $N(1)$.\\
	 Но мы знаем, что максимально возможное значение нормы достигается тогда, когда элемент -- обратимый. Тогда, так как норма любого ненулевого элемента равна $N(1)$, значит любой такой элемент -- обратим. Получается что наше кольцо -- поле. Но у нас евклидово кольцо. ПРОТИВОРЕЧИЕ.\\
	 Значит $N$ принимает бесконечное число значений.
	 \begin{flushright}
	 	\textbf{ч.т.д.}
	 \end{flushright}
	 
		
\end{document}
